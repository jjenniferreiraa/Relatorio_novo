\chapter{Metodologia Utilizada}

\setlength{\headheight}{50pt}

Para o desenvolvimento deste projeto, os estudos e desenvolvimentos previstos serão divididos em quatro macroetapas:

\begin{enumerate}

	\item[\textbf{S1}] Estudos e desenvolvimentos relacionados à avaliação de parâmetros de capacidade de carga em solos para projeto de início de poço;Planejamento das atividades e mobilização da equipe

	\item[\textbf{S2}] Estudos e desenvolvimentos relacionados ao projeto de sistema de revestimento condutor e superfície;

	\item[\textbf{S3}] Estudos numéricos para avaliação integrada
	de sistemas de início de poço

	\item[\textbf{S4}] Estudos e desenvolvimentos relacionados a geomecânica de poços em rochas salinas
\end{enumerate}

Essas etapas são indicadas nas atividades a serem desenvolvidas por meio das siglas S1, S2, S3 e S4, respectivamente. Cada etapa é composta por microetapas explicitadas a seguir:

\begin{enumerate}
\item[\textbf{S1.I}] Geração automática de folha executiva de início de poço.

A presente etapa visa o desenvolvimento de metodologia para construção de
um sistema para registro das etapas de início de poço visando a quantificação de demanda de compra/reserva de tubos e equipamentos da
fase de início de poço e consequente integração ao sistema de gerenciamento Cronoweb Materiais;

\item[\textbf{S1.II}] Estudo e desenvolvimento de metodologias para cálculo de capacidade de carga em solos considerando efeitos de adensamento
térmico (poços produtores).

O objetivo dessa etapa é o desenvolvimento e implementação de modelos para avaliar a variação da capacidade de carga em solos usados para assentamento de estruturas de início de poços e sujeitos ao efeito de adensamento térmico, especialmente em poços produtores.

\item[\textbf{S1.III}] Incorporação da influência na capacidade de carga do solo de efeitos de vibração do motor no condutor jateado.

A presente etapa consiste no desenvolvimento de metodologias para incorporação dos efeitos de vibração do motor de fundo empregado em operações de jateamento na capacidade de carga do solo para assentamento do sistema de revestimento condutor;

\item[\textbf{S1.IV}] Cálculo incremental da capacidade de carga do solo ao longo da profundidade com presença de zonas de intercalação de argila/areia.

Nesta etapa, visa-se o desenvolvimento de metodologia para levar em consideração diferentes camadas de argila e areia para avaliação de integridade de sistema de início de poço;

\item[\textbf{S1.V}] Estudos para desenvolvimento e atualização do modelo de cravabilidade em Base Torpedo.

A presenta etapa visa o aprimoramento de metodologias para avaliação de profundidade de cravação em solos cujo início de poço seja realizado por meio de lançamento de base torpedo.

\item[\textbf{S1.VI}] Atualizações na modelagem de solos considerando incertezas dos ensaios.

A presente etapa visa a incorporação de abordagens estatísticas na modelagem de solos, com posterior avaliação da capacidade de carga em solos considerando incertezas calculadas/mensuradas durante a realização de ensaios geotécnicos (como CPTu, por exemplo);

\item[\textbf{S1.VII}] Estudos para aplicação de técnicas de aprendizado de máquina na modelagem de solos.

A presente etapa objetiva a aplicação de técnicas de inteligência artificial para classificação e estimativa da capacidade de carga do solo;

\item[\textbf{S1.VIII}] Atualização dos recursos e tecnologias computacionais utilizados no sistema SEST SOLOS, incluindo atualização da versão da linguagem de programação Python, com implementação da arquitetura baseada em serviços e contêineres, além de uma mudança no framework para desenvolvimento web;

\item[\textbf{S2.I}] Análise estrutural e de estabilidade geotécnica de condutores cravados.

Nessa etapa, serão realizados desenvolvimentos ligados à avaliação de integridade estrutural conjunta solo-revestimento de sistemas condutores instalados por meio de cravação. Também serão considerados: a) Desenvolvimento de estudos e códigos computacionais para modelar o processo de cravação de condutores; b)
Otimização de instalação de condutor cravado, com desenvolvimento de metodologias para otimização do processo de assentamento de revestimento condutor cravado e consequente minimização de tempo de operação de cravação e maximização de ganho de resistência mecânica;

\item[\textbf{S2.II}] Aprimoramento de modelo em elementos finitos para interação sistema solo-condutor

As seguintes atividades são esperadas: i) alteração do elemento finito usado no SIMCON para avaliar rotação na cabeça de poço e integração de carregamentos laterais no projeto; ii) Elemento customizável preenchido pela interface web para inserção de novas curvas P-x e T-z decorrentes das medições;

\item[\textbf{S2.III}] Aprimoramento de desgaste em condutor.

Na presente fase, objetiva-se a integração completa com o sistema SIMWEAR, incluindo a avaliação de perfilagens e caracterização estatística do desgaste, para simular a influência do desgaste na integridade estrutural do conjunto solo-revestimento

\item[\textbf{S2.IV}] Aprimoramento de modelo mecano-fiabilístico com base em dados de solos e cenários de variações na fabricação de revestimentos.

Na corrente fase, está previsto o aprimoramento do modelo de avaliação de risco de falha em sistemas de início de poço considerando as
variabilidades de parâmetros estatísticos de resistência do condutor e parâmetros de resistência de solos;

\item[\textbf{S2.V}]  Atualização e geração automática de relatório de projeto de início de poço.

A presenta etapa visa a atualização de geração de relatório de início de poço com base no método específico empregado para dimensionamento;

\item[\textbf{S2.VI}]  Atualização dos recursos e tecnologias computacionais utilizados no sistema SIMCON, incluindo atualização da versão do Python,
aprimoramento da coordenação da arquitetura baseada em contêineres e mudança no framework para desenvolvimento web

\item[\textbf{S3}]  Estudos numéricos para otimização de instalação de condutor.

Nessa macroatividade, serão realizadas simulações numéricas baseadas
no método dos elementos finitos e em métodos meshless, a exemplo do método dos pontos materiais (MPM) objetivando a otimização de
parâmetros de condutores. Serão considerados os métodos de instalação por cravação e jateamento.

\item[\textbf{S4.I}] Aprimoramento de simulador de fluência salina

Esta etapa se refere à manutenção e incorporação de melhorias do simulador de
fluência salina SEST SAL, especificamente em relação à otimização do código através de duas frentes principais: a) revisão do código em
busca de otimizar seu funcionamento, e b) utilização de modelos axissimétricos híbridos compostos por trechos unidimensionais e
bidimensionais

\item[\textbf{S4.II}] Incorporação de lei de fluência primária e terciária no simulador SEST SAL

Propõe-se nessa etapa a realização de estudos e
desenvolvimentos para identificar uma lei de fluência primária para rochas salinas, e então calibrar seus coeficientes para rochas salinas
brasileiras. Este item prevê também o estudo e a incorporação de uma lei de fluência terciária ao simulador SEST SAL, a qual deve
representar o aumento na taxa de deformação e a ocorrência de dano na resposta a longo termo do material. Assim, esse item é focado no
estudo e experimentação de leis de fluência terciária.

\item[\textbf{S4.III}] Redefinição de elementos devido ao repasse de broca em rocha salina em modelos 2D

Este item envolve o emprego de técnicas de  remalhamento para tornar a operação de repasse mais eficiente numericamente. É proposto elaborar e implementar uma estratégia para remover apenas uma parte do elemento, mantendo seus campos de tensões e deformações inalterados.

\item[\textbf{S4.IV}]  Elaboração de um módulo computacional para recomendação de repasses e do peso do fluido de perfuração em rochas salinas

É proposto neste item inserir uma aba na ferramenta computacional onde automaticamente se experimente diferentes valores de peso de
fluido, a partir de um intervalo de valores por exemplo, fornecendo ao usuário informações acerca do número de repasses necessários para os diferentes pesos de fluido considerados. Essa ferramenta deve servir de apoio a decisão sobre a estratégia a ser adotada na perfuração,  acelerando o processo de experimentação.

\item[\textbf{S4.V}] Elaboração de relatório automático no sistema SEST SAL

Este item trata da elaboração automática de um relatório no formato
definido pela Petrobras com os resultados pertinentes calculados pelo módulo SEST SAL, possibilitando reduzir o trabalho de gerar e
elaborar gráficos e tabelas e organizá-los em um documento. Pretende-se agilizar o processo de emissão do projeto de perfuração,
incrementando, também, iniciativas de auditoria e revisão do mesmo

\item[\textbf{S4.VI}]  Manutenção e melhorias no módulo integrado ao PWDA

Serão realizadas aprimoramentos no simulador SEST SAL com integração ao PWDA, com consideração da lei de fluência primária e terciária definida no presente projeto.

\item[\textbf{S4.VII}] Modelagem do Leak off Test em regiões com rochas salinas

Esse item prevê a modelagem do comportamento de rochas salinas submetidas ao efeito de execução do Leak off Test, incluindo a fase de produção posterior. Haverá Implementação de Leak off Test no simulador de APB. É proposto inserir no simulador de APB já desenvolvido pelo LCCV, a modelagem do LOT. Será desenvolvida uma etapa
anterior a representação do APB na fase de produção, onde haverá a injeção de fluido em um anular específico. Serão implementadas também ferramentas para apresentar os resultados de interesse associados ao LOT. A partir de dados fornecidos pela Petrobras, será feito um estudo comparativo buscando entender o comportamento das formações envolvidas, e levantando possíveis fenômenos (como dano, fratura, porosidade, etc) que expliquem o comportamento observado em campo. A partir do estudo realizado, serão estudadas e
implementadas equações constitutivas adequadas para as formações, de forma a modelar os fenômenos observados.

\item[\textbf{S4.VIII}] Incorporação do simulador computacional LOT ao POÇO WEB

Esse item se refere à incorporação da ferramenta computacional desenvolvida ao POÇO WEB, tornando-a facilmente acessível aos potenciais usuários. Essa incorporação envolve a elaboração de interface para entrada de dados, simulação e apresentação de resultados.

\item[\textbf{S4.IX}]  Calibração de modelo de fluência salina com retroanálise de prisão de coluna de perfuração

Na presente atividade serão desenvolvidas estratégias para calibração de modelos de rochas salinas a partir de dados de falhas operacionais com ocorrência de prisão de coluna de perfuração durante a construção da fase salina.

\item[\textbf{S4.X}]  Atualização dos recursos e tecnologias computacionais utilizados no sistema SEST SAL, incluindo atualização da versão do python, aprimoramento da coordenação da arquitetura baseada em contêineres, e mudança no framework para desenvolvimento web;

\end{enumerate}
