\section{Considerações Iniciais}

A necessidade de explorar áreas cada vez mais profundas levou ao aprimoramento das técnicas de perfuração de poços. Entre os principais métodos utilizados para iniciar poços offshore, destacam-se o jateamento e a cravação. O jateamento, em particular, é altamente eficiente na instalação do revestimento condutor em sedimentos de águas profundas, que geralmente apresentam baixa consolidação e diagênese reduzida. Esse método minimiza falhas ao preservar a integridade das formações geológicas superficiais, frequentemente frágeis. Por essa razão, o jateamento é amplamente adotado como a técnica predominante na perfuração offshore (\cite{kan2018field}; \cite{akers2008jetting}).

Para o desenvolvimento deste modelo, foram adotadas diversas metodologias com o objetivo de otimizar os resultados da instalação do revestimento condutor por meio do método de jateamento. Nesse contexto, foram desenvolvidos modelos para compreender o comportamento do solo em condições específicas, utilizando o modelo viscoplástico de Herschel–Bulkley para caracterizar o comportamento reológico do solo marinho argiloso quando submetido às forças de cisalhamento exercidas pelo jato de fluido de perfuração. A seguir, são apresentados os desenvolvimentos metodológicos e computacionais adotados nesta etapa do projeto.



\section{Fundamentação e Evolução da Modelagem Numérica}

A modelagem computacional da instalação do revestimento condutor por jateamento vem sendo progressivamente adotada como alternativa às abordagens puramente empíricas, sobretudo devido às limitações operacionais, logísticas e econômicas associadas a ensaios em escala real. O avanço das ferramentas de Dinâmica dos Fluidos Computacional (CFD) possibilitou representar, com maior nível de detalhe, a interação entre fluido de perfuração, solo marinho e estrutura, permitindo avaliar fenômenos que não são diretamente observáveis em campo (\cite{kan2018field}; \cite{gomes2022modelling}).

As simulações foram conduzidas no software ANSYS Fluent 19.2, amplamente consolidado em aplicações de escoamentos multifásicos e problemas com malha dinâmica. A escolha dessa plataforma permitiu a implementação de modelos reológicos, o acompanhamento explícito das fases e a representação do movimento vertical do condutor ao longo do tempo.

O domínio computacional bidimensional adotado nesta etapa possui 90 m de altura e 80 m de largura, conforme ilustrado na Figura A1. Em estudos anteriores, domínios de menor extensão lateral mostraram-se suscetíveis à influência das condições de contorno, afetando a dissipação de pressão e a evolução do campo de velocidades. Assim, a ampliação do domínio foi necessária para minimizar efeitos numéricos artificiais e garantir que os resultados obtidos refletissem predominantemente os fenômenos físicos associados ao jateamento, e não restrições impostas pelas fronteiras do modelo.

O revestimento condutor foi modelado com 50 m de comprimento e diâmetro externo de 36 polegadas, enquanto a broca apresentou diâmetro de 17,5 polegadas. O bit stick-out, definido como a distância entre a broca e a base do condutor, foi fixado em 0,3 m, e a elevação inicial do conjunto em relação ao fundo do mar foi estabelecida em 0,5. Adicionalmente, foi definida uma linha de monitoramento ao longo do eixo central do condutor, conforme ilustrado na Figura A.1, com o objetivo de rastrear a evolução espacial e temporal de propriedades representativas do processo de jateamento.


\begin{figure}[H]
	\centering
	\includegraphics[width=0.8\linewidth]{jateamento/Jateamento_01.pdf}
	\caption{Configuração geométrica e sistema de monitoramento. (a) Dimensões do domínio de simulação; (b) Linha central de monitoramento. Fonte: Autores (2025)}
	\label{fig:Jateamento_01}
\end{figure}

Com base em estudos prévios reportados na literatura, a fração volumétrica inicial do solo foi assumida como VoF = 1 (\cite{guo2022evaluation}), representando um material completamente saturado e contínuo. Essa hipótese mostrou-se adequada para a análise do processo de jateamento em solos argilosos moles, além de contribuir para a estabilidade numérica das simulações multifásicas.
As propriedades físicas e reológicas do solo utilizadas no modelo computacional foram definidas com base em valores típicos reportados na literatura para argilas marinhas moles e são apresentadas na Tabela A1.


\begin{table}[H]
	\centering
	\begin{tabular}{|l|l|l|}
		\hline
		& \textbf{Solo} & \textbf{Água} \\ \hline
		\textbf{Peso Específico (kg/m³)}               & 2300          & 998.3         \\ \hline
		\textbf{Viscosidade (Pa.s)}                    & -             & 0.001         \\ \hline
		\textbf{Viscosidade de escoamento(Pa.s)}       & -             & -             \\ \hline
		\textbf{Tensão de escoamento (Pa)}             & 40000         & -             \\ \hline
		\textbf{Índice de comportamento do fluido (n)} & 0.1          & -             \\ \hline
		\textbf{Índice de consistência (Pa.s)}         & 188310        & -             \\ \hline
		\textbf{Taxa de escoamento crítica (1/s)}    & 5.11          & -             \\ \hline
	\end{tabular}
	\caption{Parâmetros reológicos dos fluidos. Autores: \citet{gomes2022modelling} \citet{pacheco2020numerical}; \citet{salam2019enhancement}}
	\label{fig:Jat_tabelaprop}
\end{table}

No contexto da dissipação das tensões induzidas pelo jato, os resultados indicaram que a distribuição e a magnitude das tensões no solo são fortemente dependentes do nível de vazão imposto, conforme observado em estudos clássicos de jateamento em solos moles (\cite{beck1991reliable}). Conforme ilustrado na Figura A.2, o perfil de descida ao longo de 30 metros do condutor evidência regiões de intensa fluidização próximas à broca, seguidas por zonas onde ocorre dissipação gradual das tensões ao longo da interface solo-condutor Figura A3.


\begin{figure}[H]
	\centering
	\includegraphics[width=0.8\linewidth]{jateamento/Jateamento_02.pdf}
	\caption{Evolução da fração volumétrica do solo (VoF) durante a descida do revestimento condutor: (a) estado inicial em 0 s; (b) penetração intermediária em 75 s; e (c) profundidade final em 150 s. Fonte: Autores (2025)}
	\label{fig:Jateamento_02}
\end{figure}

\begin{figure}[H]
	\centering
	\includegraphics[width=0.8\linewidth]{jateamento/Jateamento_03.pdf}
	\caption{Imagens da dissipação de pressão em diferentes instantes da operação: (a) 5 minutos; (b) 10 minutos; (c) 15 minutos; e (d) 20 minutos. Fonte: Autores (2025)}
	\label{fig:Jateamento_03}
\end{figure}

Embora esse mixed soil possa apresentar, ao longo do tempo, um ganho percentual elevado de resistência devido ao efeito de setup, seu valor final de resistência ao cisalhamento tende a permanecer significativamente inferior ao do solo intacto, uma vez que o processo se inicia a partir de um estado de resistência muito reduzida (\cite{beck1991reliable}). Tal comportamento possui implicações diretas na estabilidade do condutor e na integridade do sistema durante e após a instalação.

Entretanto, os resultados do modelo CFD indicaram a migração da água para o interior do fluido viscoso que representa o solo, que não corresponde plenamente ao mecanismo físico esperado em solos reais. Essa limitação decorre da representação do solo como um fluido não newtoniano homogêneo, o que pode superestimar a mobilidade relativa da fase líquida no interior da matriz sólida. Dessa forma, esse aspecto será objeto de investigação em etapas futuras do trabalho, visando o aprimoramento da representação física do solo.


\section{Modelagem em desenvolvimento}

Esta seção apresenta a etapa atual de desenvolvimento da modelagem computacional da instalação do revestimento condutor por jateamento. Diferentemente de abordagens anteriores mais simplificadas, a estratégia adotada nesta fase busca aumentar o grau de realismo do modelo por meio da incorporação de funções definidas pelo usuário (User-Defined Functions – UDFs), permitindo a representação explícita de condições operacionais extraídas de dados reais de campo.


\subsection{Metodologia atual}

A metodologia atualmente em desenvolvimento baseia-se na utilização de três UDFs implementadas em linguagem C++, integradas ao solver do ANSYS Fluent, como ilustra na figura A.4, com o objetivo de representar de forma mais fiel a dinâmica do processo de jateamento e da instalação do revestimento condutor. Essas funções foram empregadas para controlar a descida do condutor, a vazão do jato e a resposta reológica do solo, ampliando a complexidade e a aderência física do modelo numérico.

\begin{figure}[H]
	\centering
	\includegraphics[width=0.6\linewidth]{jateamento/Jateamento_04.pdf}
	\caption{Esquema de implementação das funções definidas pelo usuário (UDFs) no ANSYS Fluent.}
	\label{fig:Jateamento_04}
\end{figure}

A cinemática de descida do condutor foi ajustada de acordo com a duração real de um processo de jateamento de Marlim, conforme descrito no Relatório do Jateamento do Revestimento na Figura A.4. Em vez de impor uma velocidade constante arbitrária, a UDF responsável pelo movimento vertical prescreve a descida do condutor em função do tempo total de jateamento observado em campo.

\begin{figure}[H]
	\centering
	\includegraphics[width=0.8\linewidth]{jateamento/Jateamento_05.pdf}
	\caption{Trecho do Relatório de Jateamento do Revestimento referente à duração do processo de jateamento.}
	\label{fig:Jateamento_05}
\end{figure}

De maneira análoga, a vazão do jato utilizada nas simulações também foi extraída de dados operacionais de um processo real de jateamento de marlim. Essa vazão é imposta por meio de uma UDF específica, possibilitando a aplicação de condições de contorno coerentes com a prática operacional e evitando a adoção de valores puramente paramétricos ou idealizados.

\begin{figure}[H]
	\centering
	\includegraphics[width=0.7\linewidth]{jateamento/Jateamento_06.pdf}
	\caption{Tabela extraída do Relatório de Jateamento do Revestimento do campo de Marlim contendo os valores operacionais de vazão do jato.}
	\label{fig:Jateamento_06}
\end{figure}

A terceira UDF foi utilizada para incorporar ao modelo reológico do solo o valor de SUTT (Shear Undrained Ultimate Strength) obtido a partir do teste GT-34, fornecido como dado de entrada para o presente estudo. Esse parâmetro foi acoplado diretamente ao modelo viscoplástico de Herschel-Bulkley, aumentando a consistência geotécnica da modelagem.

\begin{figure}[H]
	\centering
	\includegraphics[width=0.7\linewidth]{jateamento/Jateamento_07.pdf}
	\caption{Tabela de resultados do ensaio geotécnico GT-34 contendo os valores de SUTT.}
	\label{fig:Jateamento_07}
\end{figure}


\subsection{Resultados atuais}

Os resultados atualmente obtidos correspondem à simulação da descida do revestimento condutor ao longo de 7,5 metros, representando a fase inicial do processo de jateamento. Essa etapa permitiu avaliar a resposta do solo às condições operacionais impostas, bem como a influência direta de parâmetros geotécnicos incorporados ao modelo por meio das funções definidas pelo usuário. A Figura A.8 apresenta a evolução da descida do condutor ao longo do tempo durante essa fase inicial da simulação.

\begin{figure}[H]
	\centering
	\includegraphics[width=0.7\linewidth]{jateamento/Jateamento_08.pdf}
	\caption{Evolução temporal da profundidade de descida do revestimento condutor durante a fase inicial do processo de jateamento, correspondente aos primeiros 7,5 m de descida. Fonte: Autores (2025)}
	\label{fig:Jateamento_08}
\end{figure}

A análise concentrou-se na evolução da resistência do solo e na variação da viscosidade aparente, ambas diretamente associadas à modificação do valor de SUTT acoplado ao modelo viscoplástico de Herschel–Bulkley e ao nível de vazão imposto aos jatos de perfuração. Observou-se que alterações nesses parâmetros resultam em mudanças significativas no comportamento reológico do solo, afetando sua resposta ao cisalhamento imposto tanto pelo jato quanto pelo avanço do condutor. A Figura A.9 ilustra a pressão do solo ao longo do tempo.


\begin{figure}[H]
	\centering
	\includegraphics[width=0.7\linewidth]{jateamento/Jateamento_09.pdf}
	\caption{Evolução temporal da pressão do solo durante a fase inicial do processo de jateamento. Fonte: Autores (2025)}
	\label{fig:Jateamento_09}
\end{figure}

Adicionalmente, foi observada a entrada da fase associada ao solo no interior do condutor ao longo da simulação. Esse comportamento sugere a ocorrência de um processo de arraste do material fluidizado para dentro do revestimento, possivelmente relacionado ao nível de vazão imposto no jato. Diante desse resultado, estão sendo conduzidas investigações adicionais com o objetivo de avaliar estratégias para mitigar ou evitar a entrada do solo no interior do condutor, incluindo ajustes nos níveis de vazão, na formulação das funções definidas pelo usuário e na representação numérica do acoplamento entre as fases.

\section{Considerações finais}

O presente trabalho apresentou o estágio atual de desenvolvimento de uma metodologia computacional para a modelagem da instalação de revestimento condutor por jateamento, com foco na incorporação progressiva de maior realismo físico e operacional ao modelo numérico. A utilização de funções definidas pelo usuário permitiu integrar ao ambiente de simulação dados reais de operação e de ensaios geotécnicos, elevando o nível de complexidade e a representatividade do problema analisado.

Apesar dos avanços alcançados, reconhece-se que a modelagem ainda se encontra em desenvolvimento. Limitações inerentes à representação do solo como um meio contínuo viscoplástico permanecem, especialmente no que diz respeito à migração da água no interior do material modelado. Esses aspectos serão objeto de investigações, com vistas ao aprimoramento da representação física do solo e à ampliação do escopo das simulações.
