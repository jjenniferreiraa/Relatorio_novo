%Resumo do projeto. Normalmente esse resumo já foi escrito durante o processo de elaboração do projeto. No caso da petrobras, esse texto pode ser retirado diretamente do SIGITEC.
\setlength{\headheight}{50pt}

\begin{abstract}

Este projeto visa o desenvolvimento de métodos e ferramentas computacionais de geomecânica salina e geotécnica de poços para avaliação de integridade na perfuração em rochas salinas e na gestão de integridade do sistema solo-revestimento em todo ciclo produtivo do poço. A metodologia de desenvolvimento deste projeto contempla as seguintes macroetapas: a) estudos e desenvolvimentos relacionados à avaliação de parâmetros de capacidade de carga em solos para projeto de início de poço (instalado por cravação, jateamento ou perfurado e cimentado); 2) estudos e desenvolvimentos relacionados ao projeto de sistema de revestimento condutor e superfície; 3) estudos numéricos para avaliação integrada solo-revestimento e otimização de parâmetros de dimensionamento; 4) estudos e desenvolvimentos relacionados a geomecânica de poços em rochas salinas e na realização de LOT. Este projeto contribui de forma significativa no desenvolvimento de operações de início de poço, perfuração em rochas salinas e na gestão de integridade ao longo da vida operacional, visando atendimento de normativos de segurança operacional.O presente relatório parcial 1 visa a apresentação das atividades técnicas realizadas no período de 01/07/2023 a 29/02/2024.

\end{abstract}