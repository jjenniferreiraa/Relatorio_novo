\section{Considerações Iniciais }

A etapa inicial da perfuração de um poço de petróleo ocorre com a instalação do revestimento condutor, ou estrutural, no solo. Conforme sugere, a função deste tubo é estabilizar as paredes do poço e fornecer suporte aos equipamentos de cabeça e aos revestimentos das fases subsequentes. A sua instalação o varia com a localização do poço (\textit{onshore} ou \textit{offshore}) e as características geotécnicas do solo.  

No cenário \textit{offshore}, a cravação é uma alternativa recorrente para o início de poço. O processo que levará à execução dessa atividade consiste de certas etapas, dentre elas, o ajuste de parâmetros de projeto, muitas vezes, realizado com base em ensaios experimentais. No entanto, em lâminas d’água elevadas, a execução desses ensaios tende a ser arriscada e dispendiosa, estendendo cronogramas. Para mitigar custos e riscos, estudos numéricos apoiados em modelos validados e calibrados oferecem uma representação confiável da instalação do condutor no solo.

O presente apêndice reúne os resultados da modelagem numérica da cravação de um revestimento condutor, considerando um domínio em dimensões e dados reais. O desenvolvimento foi suportado por dados de solo do campo Papa Terra, abrangendo relatórios de  instalação por martelamento e resultados de sondagens CPTu realizados nesse campo.

Como é sabido, problemas de impacto deste tipo envolvem grandes deformações e, ao serem modelados numericamente, resultam em grandes distorções de malha computacional se solucionados por métodos numéricos baseados na mecânica do contínuo.  Como alternativa, vêm ganhando destaque na comunidade geotécnica os métodos numéricos baseados em partículas, com destaque para o Método dos Pontos Materiais (MPM), particularmente adequado para problemas com grandes deformações. Assim, o MPM se apresenta como uma abordagem promissora para a modelagem desta operação.


\section{Modelagem numérica}

Para atender ao desafio proposto, a análise numérica foi realizada no \textit{software} \textit{open source} ANURA3D, que utiliza o Método dos Pontos Materiais para processar os cálculos. Na sequência, apresentam-se, de forma sucinta, as formulações e os algoritmos adotados, bem como a descrição da geometria, das condições de contorno e das propriedades dos materiais.

\subsection{Método dos Pontos Materiais}

O Método dos Pontos Materiais (MPM) é uma abordagem híbrida Euleriana-Lagrangiana que combina as vantagens de métodos numéricos baseados em partículas e baseados em malha. Nesse arcabouço, o domínio computacional é discretizado por uma malha Euleriana de fundo fixa. Ao mesmo tempo, os corpos materiais são representados por partículas Lagrangianas (pontos materiais) que transportam todas as propriedades físicas e constitutivas (massa, densidade, tensões, deformações, etc.) e se deslocam através da malha. Essa dupla caracterização permite ao MPM lidar naturalmente com grandes deformações, evitando os problemas de distorção de malha inerentes aos métodos puramente Eulerianos (\citet{yerro2022modelling}; \citet{nguyen2023material}). A Figura \ref{fig:MPM} mostra um breve esquema do ciclo de simulação numérica utilizando o MPM.

\begin{figure}[H]
	\centering
  	\includegraphics[width=1\linewidth]{cravacao/MPM2.pdf}
	\caption{Algoritmo de resolução do MPM. Fonte: Autores (2026)}
	\label{fig:MPM}
\end{figure}

O método impõe a conservação da quantidade de movimento, conforme expresso na Eq.\eqref{eq:quantidade_movimento}. A integração numérica prossegue pelas etapas: cálculo das funções de forma para todos os pontos materiais ao longo da malha de fundo e transferência das propriedades de massa e quantidade de movimento das partículas para os nós (\ref{fig:MPM}a); solução das equações de quantidade de movimento na malha (Eqs.\eqref{eq:forca_interna} e \eqref{eq:forca_externa}) (\ref{fig:MPM}b); e projeção das velocidades e posições nodais atualizadas de volta aos pontos materiais (\ref{fig:MPM}c). Onde $\sigma_p$ é a tensão de Cauchy; $\nabla N_I(\mathbf{x}_p)$ é o gradiente da função de forma; $V_p$ e $\rho_p$ são, respectivamente, o volume e a densidade do ponto material; $\mathbf{b}$ é a força de corpo; e $\mathbf{f}_I^{int}$, $\mathbf{f}_I^{ext}$ são as forças nodais internas/externas. Ao final de cada passo de tempo, a malha é reestruturada para a posição inicial e as partículas permanecem na posição deslocada, prevenindo distorções excessivas nos elementos (\citet{martinelli2020numerical}, \citet{yost2022}, \citet{fu2024material}).

\begin{equation}
	\rho \frac{dv}{dt} = \nabla \cdot \sigma + \rho \mathbf{b}
	\label{eq:quantidade_movimento}
\end{equation}

\begin{equation}
	f_I^{\text{int}} = - \sum_p \sigma_p \,\nabla N_I(\mathbf{x}_p)\, V_p
	\label{eq:forca_interna}
\end{equation}

\begin{equation}
	f_I^{\text{ext}} = \sum_p N_I(\mathbf{x}_p)\,\rho_p\,\mathbf{b}\, V_p
	\label{eq:forca_externa}
\end{equation}

\subsection{\textit{Two-phase single point formulation:} modelagem para problemas de interação solo-água}

A formulação \textit{Two-phase single point}, em modo explícito no MPM, é capaz de descrever como ocorre a interação solo-água, levando-se em consideração a geração e a dissipação de pressão de poros mediante a descida da estrutura no solo, assim como a propagação das ondas dinâmicas em ambas as fases \citet{alkafaji2013}. Nesta formulação todas as fases (por exemplo, esqueleto sólido, água e, eventualmente, ar) são representadas por um conjunto único de pontos materias. Os pontos materiais se movem juntamente com o movimento sólido (descrição lagrangiana), enquanto os fluidos são descritos usando um referencial euleriano. Essa representação dos meios multifásicos é conveniente para muitas aplicações geotécnicas, mas não considera a separação física entre água livre e esqueleto sólido, como é no caso da aplicação no presente modelo \citet{ceccato2024simulating}. As equações de governo são conservação de massa, conservação de momento e um modelo constitutivo capaz de prever a solução do sistema, descritas as equações 1 a 5, conforme descrito por \cite{martinelli2022explicit}:


\begin{equation}
	\rho_L \textbf{a}_L = \nabla \cdot (p_L\textbf{}I)+  \rho_L \textbf{b}-\frac{n\mu_L}{\kappa_L} (\nu_L - \nu_S)
\end{equation}

\begin{equation}
	(1-n)\rho_S\textbf{a} + n\rho_L\textbf{a} = \nabla \cdot (\sigma'+p_L\textbf{I})+\rho_m\textbf{b}
\end{equation}

\begin{equation}
	\frac{dn}{dt}=(1-n)(\nabla \cdot \nu_S)
\end{equation}

\begin{equation}
	\dot{P} = \frac{K_L}{n}[(1-n)\nabla \cdot \nu_S + n\nabla \cdot \nu_L]
\end{equation}

\begin{equation}
	\dot{\sigma} = D\dot{\varepsilon}_S - \Omega\sigma' - \sigma'\Omega - \dot{\varepsilon}_{\nu,S} \sigma'
\end{equation}.

Onde, $n$ é a porosidade; $\rho_S$ e $\rho_L$ são as densidades dos grãos e da fase líquida, respectivamente; $\rho_m = (1-n)\rho_S + n\rho_L$ é a densidade da mistura; \textbf{I} é o tensor identidade; $\nu_S$ e $\rho_L$ são, respectivamente, as velocidades das fases sólidae líquida; $\sigma'$ é o tensor de tensões; $K_L$ é o módulo volumétrico da água pura; $\textbf{b}$ é o vetor de força do corpo; $\textbf{D}$ é a matriz de rigidez; $\cdot{\sigma'}$ e $\dot{\varepsilon}$ são, respectivamente, as taxas de tensão e de deformação da fase sólida; $\Omega$ é o tensor de rotação e $\varepsilon_{\nu,s}$ é o incremento de deformação volumétrica \citep{martinelli2022explicit}. As equações 6 e 7 contêm, respectivamente, a forma discretizada das equações de balanço 1 e 2 para um nó ativo genérico da malha computacional.

\begin{equation}
	\bar{M}_{L,i}=\bar{f}^{ext}_{L,i} - \bar{f}^{int}_{L,i} + \bar{f}^d_i
\end{equation}

\begin{equation}
	\bar{M}_{S,i}a_{S,i} + \bar{M}_La_{L,i} = \bar{f}^{ext}_i - \bar{f}^{int}_i 
\end{equation}


Nas quais, $\bar{M}_{S,i}$, $\bar{M}_{L,i}$, $\bar{f}^{ext}_{L,i}$, $\bar{f}^{int}_{L,i}$, $\bar{f}^d$ são os valores nodais, respectivamente, para: a matriz de massa das fases sólida e líquida; os vetores de força externa, de força interna e de força de arrasto. Estas forças são dependentes do número de elementos ao redor do nó, do número de pontos materiais em cada elemento de malha, da função de forma calculada para a posição de ponto material e da força gravitacional.

\subsection{Algoritmo de contato}
Na Engenharia Geotécnica, é comum lidar com problemas que envolvam interação solo-estrutura. Quando um deslizamento por atrito ocorre na superfície de contato entre estes corpos, necessita-se de um algoritmo específico, que permita o movimento relativo entre eles. O algoritmo de contato utilizado foi orginalmente desenvolvido por \citet{bardenhagen2001improved}, que o formulou para modelar tanto a separação entre os corpos quanto o contato com deslizamento por atrito entre eles. Posteriormente, este algoritmo foi extendido a contatos adesivos por \citet{alkafaji2013} \citep{martinelli2021investigation}, como é o caso dos solos coesivos não drenados. A vantagem desse algoritmo é que ele detecta a superfície de contato automaticamente, ou seja, não é necessário definir nenhum elemento especial na interface entre os corpos \citep{alkafaji2013}.

Inicialmente, o efeito do contato é realizado pela correção da diferença de velocidade entre os corpos: seja um sistema com dois corpos A e B em contato de deslizamento, cujas velocidades prescritas individuais (${\nu}_{k,a}$ e ${\nu}_{k,b}$) e a velocidade combinada do sistema $({\nu}_{k,s})$ são resolvidas através das equações de movimento para um nó de contato $k$. Tomando a velocidade prescrita do corpo A num instante de tempo $t$+$\Delta t$:

\begin{equation}
	{\begin{matrix}
			(\nu_{k,A}^{t+\Delta t}-\nu_{k,S}^{t+\Delta t}) \cdot n_k^t>0 
			\\
			(\nu_{k,A}^{t+\Delta t}-\nu_{k,S}^{t+\Delta t}) \cdot n_k^t<0
		\end{matrix})}
\end{equation}
	
	\subsection{Algoritmo do corpo rígido}
	
	O esquema de integração explícita é condicionalmente estável. O tamanho do passo de tempo para uma solução estável decresce com o aumento da rigidez do material e com a diminuição do tamanho do elemento da malha computacional \citep{martinelli2020numericalm}. A fim de reduzir o custo computacional, o condutor é modelado como um corpo rígido através do algoritmo desenvolvido por \citet{zambrano2020numerical}. Esta aproximação desconsidera a propagação de onda na estrutura e torna-se válida porque a rigidez do aço do condutor é muito maior se comparada à rigidez do solo \citep{galavi2019}.
	
	\subsection{Malha móvel}
	
	Com o intuito de preservar os elementos de malha ao redor da estrutura da deformação imposta, utiliza-se o procedimento de malha móvel. A malha móvel se aproveita do fato da malha computacional não armazenar informações permanentes, já que elas são transmitidas aos pontos materiais Cecatto (2022). Este procedimento consiste basicamente em ajustar a malha adjacente à estrutura ao seu movimento após cada passo de tempo, assegurando que a sua superfície coincida com os elementos de fronteira e, com isso, os elementos manterão a mesma forma durante a simulação \citep{ceccato2017numerical}: a malha móvel será a porção da malha cujos elementos não se deformarão; por outro lado, a porção da malha localizada entre a malha móvel e as restrições do domínio irá se deformar.
	
	\subsection{Histórico de desenvolvimento do modelo numérico atual}
	
	Para que a geometria do sistema seja devidamente representativo, deve-se construir um modelo numérico característico da situação real do problema proposto, considerando-se as variáveis de projeto, as restrições adotadas e o tamanho ideal do domínio, que seja suficiente para evitar efeitos de borda,  mas que tenha o menor custo computacional possível sem comprometer a qualidade dos resultados da simulação. Ao calibrar um modelo numérico, por muitas vezes, deve-se adaptar a geometria e realizar o remalhamento da malha computacional. 
	
	O modelo numérico até então desenvolvido conseguiu cravar até a metade da segunda camada de solo, totalizando cerca de 13,5 m de profundidade a partir da \textit{mudline}. Depois disso, a simulação abortou por problemas que, ainda nos dias de hoje, estão sob análise. Provavelmente, ainda são problemas numéricos relacionados às partículas nas regiões de \textit{shoulder} (o encontro entre a ponta e a base do condutor) e ponta do condutor, onde ocorrem grandes concentrações de tensão. No "\textit{shoulder}" é comum que um fenômeno numérico indesejado, a "interpenetração", ocorra. Nele duas ou mais partículas materiais ocupam o mesmo espaço simultaneamente, ou seja, suas fronteiras se sobrepõem. A interpenetração se dá, por exemplo, devido a uma deformação extrema nos corpos, é o que se observa no problema proposto, dado que há uma estrutura rígida penetrando um corpo altamente deformável. Ou, então, devido a erros numéricos, como a discretização inadequada das equações de movimento. 
	
	No caso em questão, provavelmente ocorreu porque no vértice do \textit{shoulder}, estavam constando dois vetores normais, que "confundiam" o movimento das partículas, resultando na interpenetração, conforme mostra a Figura \ref{fig:mudanca_ponta}. Um dos pontos revisados foi o remalhamento e a alteração no contato solo-estrutura.  Salienta-se que ao, realizar a suavização do \textit{shoulder}, é necessário se atentar ao refinamento da malha nesta região. Um refinamento muito excessivo pode elevar bastante o custo computacional da simulação. A Figura \ref{fig:mudanca_ponta}a ilustra a configuração da ponta deste modelo numérico, um um \textit{shoulder} mais acentuado e a ponta menos refinada. A Figura \ref{fig:mudanca_ponta}b já mostra a nova configuração desta mesma região que, embora apresente uma mudança sutil, melhorou consideravelmente os resultados.
	
	\begin{figure}[H]
		\centering
		\includegraphics[width=0.7\linewidth]{cravacao/mudanca_ponta.pdf}
		\caption{Breve descrição da transição realizada na ponta do condutor. Fonte: Autores (2026)}
		\label{fig:mudanca_ponta}
	\end{figure}

Em relação ao último resultado, o perfil de descida do condutor ao longo do tempo pode ser analisado na Figura \ref{fig:cravacao_result2}. No início da operação, durante a fase de peso próprio, a velocidade de descida do condutor reduz gradualmente até se estabilizar por volta dos 5 segundos. A partir desse ponto, iniciam-se os impactos de martelamento.
	
	\begin{figure}[H]
		\centering
		\includegraphics[width=0.8\linewidth]{cravacao/cravacao_result2.pdf}
		\caption{Profundidade alcançada após a fase de martelamento. Atualmente, o condutor está em uma profundidade de aproximadamente 13,50 metros. Fonte: Autores (2025)}
		\label{fig:cravacao_result2}
	\end{figure}
	
	Durante a fase de peso próprio, o revestimento do condutor atinge aproximadamente 9,5 metros de profundidade após 5 segundos, penetrando a primeira camada de solo e adentrando a segunda, que apresenta maior resistência. Normalmente, essa fase resulta em profundidades entre 10 e 15 metros, indicando que o valor observado está próximo do esperado.
	
Embora a simulação tenha abortado após os 13 m, a Figura \ref{fig:historico2} mostra que solo apresentou um comportamento fisicamente consistente.  Ressalta-se que uma adaptação importante foi considerar o condutor com a ponta fechada. A ponta fechada ocorre devido ao fenômeno de "embuchamento", no qual um \textit{plug} de solo argiloso se forma logo no início da cravação, ocupando a secção transversal do condutor devido à adesão da argila em suas paredes internas. Com isto, foi possível considerar o condutor como sendo uma estrutura maciça, com raio de 18" (0,4572 m). Essa medida foi necessária para melhorar a transição de malha computacional do revestimento para o ambiente externo, elevando a qualidade dos elementos de malha.
	
Para garantir a precisão do modelo na previsão da interação solo-estrutura, foi realizada uma calibração com dados operacionais, conforme ilustrado na Figura \ref{fig:cravacao_result3}.

\begin{figure}[H]
	\centering
	\includegraphics[width=0.7\linewidth]{cravacao/cravacao_result3.pdf}
	\caption{Número acumulado de impactos ao longo da profundidade alcançada somente durante a fase de penetração do martelo. Fonte: Autores (2025)}
	\label{fig:cravacao_result3}
\end{figure}

A Figura \ref{fig:cravacao_result3} compara os dados operacionais da indústria (curva azul) com os resultados simulados (curva vermelha), evidenciando um comportamento semelhante entre ambos. Ajustes futuros no modelo e refinamentos nas simplificações aplicadas podem melhorar ainda mais essa correspondência, conforme será visto mais adiante, nos resultados atuais. 
	
	
	
	\section{Simulação da cravação do revestimento condutor por martelamento}
	
	Com base nos conhecimentos adquiridos a partir das simulações anteriores, foi possível avançar para uma abordagem mais refinada, resultando no desenvolvimento de um novo modelo para a simulação da cravação do revestimento condutor por martelamento.
	
	\subsection{Metodologia}
	
	Assim, a geometria foi implementada utilizando um modelo 2D axisimétrico. O domínio foi discretizado em uma malha não estruturada composta por 4.884 elementos triangulares de três nós, 2.539 nós e 13.805 pontos materiais, distribuídos de forma que a região do solo contém seis pontos materiais por elemento, enquanto a região do condutor possui um por elemento. 
	
	A geometria completa possui 124,68 metros de altura e 29,26 metros de raio. O domínio do solo argiloso é segmentado em quatro camadas distintas: a primeira se estende até 8 metros de profundidade, a segunda até 10 metros, e a terceira e quarta atingem 17 e 25 metros, respectivamente. Para reduzir o custo computacional nas análises iniciais, o solo foi limitado a uma profundidade de 60 metros. O condutor, por sua vez, possui um diâmetro externo de 36 polegadas (~0,91 metros) e um comprimento total de 58,8 metros. A adesão da argila na seção transversal forma um tampão que veda o diâmetro interno do revestimento, permitindo que ele seja tratado como uma estrutura maciça com um peso linear de 3.470 lb/ft. Essa consideração reduziu a necessidade de refinamento excessivo na região próxima ao condutor, facilitando o processo de discretização da malha.
	
	Conforme ilustrado na Figura \ref{fig:Dominio_cracavao}, o revestimento condutor encontra-se inicialmente posicionado acima da mudline. Para melhorar a estabilidade numérica, foi implementada uma ponta no condutor, conforme abordado em estudos anteriores (\citet{ceccato2017adhesive}; \citet{galavi2017numerical}; \citet{galavi2018numerical}; \citet{martinelli2021investigation}; \citet{martinelli2022explicit}; \cite{yost2023addressing}; \citet{ceccato2024simulating}).
	
	As condições de contorno foram definidas da seguinte forma: o movimento horizontal é restrito nas laterais, o deslocamento vertical é limitado no topo, e a fronteira inferior é fixa para impedir qualquer deslocamento. O condutor pode se mover exclusivamente na direção vertical. Além disso, o procedimento de malha móvel está ilustrado na Figura \ref{fig:Dominio_cracavao}, onde a malha comprimida é restrita à região do solo, enquanto a malha móvel ocupa a parte superior, tendo o condutor como referência para seu deslocamento. O revestimento foi modelado como um corpo rígido, e a interação solo-estrutura foi representada por meio do algoritmo de contato adesivo proposto por \citet{alkafaji2013}.
	
	\begin{figure}[H]
		\centering
		\includegraphics[width=0.7\linewidth]{cravacao/Dominio_cravacao.pdf}
		\caption{Geometria em camadas, distribuição da malha e geometria do condutor de um modelo numérico 2D axissimétrico. Fonte: Autores (2024)}
		\label{fig:Dominio_cracavao}
	\end{figure}
	
	\subsubsection{Propriedades dos materiais}
	
	A definição adequada das propriedades dos materiais e dos parâmetros do modelo constitutivo é essencial para garantir a representatividade das simulações numéricas. Assim, de acordo com os dados de resistência à compressão foi possível classificar o solo como uma argila muito mole, considerando-se o critério adotado por \citet{maragon2008}.
	
	Os dados de ensaio CPTu utilizados são disponibilizados pela operadora, a partir dos quais constatou-se que o modelo constitutivo de Mohr-Coulomb seria adequado para modelar a resposta do solo. A Tabela \ref{tab:propriedades_crav} contém as propriedades dos materiais considerados para modelar o solo coesivo não drenado em questão.
	
	\begin{table}[H]
		\centering
		\footnotesize % Reduz tamanho da fonte
		\renewcommand{\arraystretch}{1.2} % Aumenta espaçamento entre linhas
		\setlength{\tabcolsep}{4pt} % Ajusta espaçamento entre colunas
		\begin{tabular}{|cccccc|}
			\hline
			\multicolumn{1}{|c|}{\multirow{2}{*}{\textbf{Propriedades}}} &
			\multicolumn{4}{c|}{\textbf{Valor}} &
			\multirow{2}{*}{\textbf{Unidade}} \\ \cline{2-5}
			\multicolumn{1}{|c|}{} &
			\multicolumn{1}{c|}{Camda 1} &
			\multicolumn{1}{c|}{Camada 2} &
			\multicolumn{1}{c|}{Camada 3} &
			\multicolumn{1}{c|}{Camada 4} &
			\\ \hline
			\multicolumn{1}{|c|}{\textbf{Classificação}} &
			\multicolumn{1}{c|}{Argila muito mole} &
			\multicolumn{1}{l|}{Argila muito mole} &
			\multicolumn{1}{l|}{Argila mole} &
			\multicolumn{1}{l|}{Argila média} &
			\multicolumn{1}{l|}{} \\ \hline
			\multicolumn{1}{|c|}{\textbf{Porosidade inicial}} &
			\multicolumn{1}{c|}{0.58} &
			\multicolumn{1}{c|}{0.58} &
			\multicolumn{1}{c|}{0.47} &
			\multicolumn{1}{c|}{0.43} &
			- \\ \hline
			\multicolumn{1}{|c|}{\textbf{Densidade}} &
			\multicolumn{1}{c|}{1,475.40} &
			\multicolumn{1}{c|}{1,687.91} &
			\multicolumn{1}{c|}{1,799.32} &
			\multicolumn{1}{c|}{1,838.82} &
			Kg/m³ \\ \hline
			\multicolumn{1}{|c|}{\textbf{Coeficiente de Poisson \linebreak efetivo}} &
			\multicolumn{1}{c|}{0.49} &
			\multicolumn{1}{c|}{0.49} &
			\multicolumn{1}{c|}{0.45} &
			\multicolumn{1}{c|}{0.40} &
			- \\ \hline
			\multicolumn{1}{|c|}{\textbf{Valor de K0}} &
			\multicolumn{1}{c|}{0.96} &
			\multicolumn{1}{c|}{0.96} &
			\multicolumn{1}{c|}{0.82} &
			\multicolumn{1}{c|}{0.67} &
			- \\ \hline
			\multicolumn{1}{|c|}{\textbf{Módulo de elasticidade efetivo}} &
			\multicolumn{1}{c|}{17,896.07} &
			\multicolumn{1}{c|}{27,823.35} &
			\multicolumn{1}{c|}{60,440.32} &
			\multicolumn{1}{c|}{83,865.38} &
			kPa \\ \hline
			\multicolumn{1}{|c|}{\textbf{Coesão efetiva}} &
			\multicolumn{1}{c|}{9,27} &
			\multicolumn{1}{c|}{36.23} &
			\multicolumn{1}{c|}{93.95} &
			\multicolumn{1}{c|}{148.92} &
			kPa \\ \hline
			\multicolumn{1}{|c|}{\textbf{Ângulo de atrito efetivo}} &
			\multicolumn{1}{c|}{36.29} &
			\multicolumn{1}{c|}{37.99} &
			\multicolumn{1}{c|}{40.35} &
			\multicolumn{1}{c|}{37.63} &
			graus \\ \hline
			\multicolumn{1}{|c|}{\textbf{Ângulo de atrito (contato)}} &
			\multicolumn{1}{c|}{0.43} &
			\multicolumn{1}{c|}{0.44} &
			\multicolumn{1}{c|}{0.47} &
			\multicolumn{1}{c|}{0.44} &
			rad \\ \hline
			\multicolumn{1}{|c|}{\textbf{Fator de adesão}} &
			\multicolumn{1}{c|}{4.64} &
			\multicolumn{1}{c|}{18.12} &
			\multicolumn{1}{c|}{46.97} &
			\multicolumn{1}{c|}{74.46} &
			kPa \\ \hline
			\multicolumn{6}{|c|}{\textbf{Água}} \\ \hline
			\multicolumn{1}{|c|}{\textbf{Densidade}} &
			\multicolumn{4}{c|}{1000} &
			Kg/m³ \\ \hline
			\multicolumn{1}{|c|}{\textbf{Permeabilidade intrínseca}} &
			\multicolumn{4}{c|}{1.0214x10-9} &
			m³/s \\ \hline
			\multicolumn{1}{|c|}{\textbf{Módulo volumétrico}} &
			\multicolumn{4}{c|}{2.15x104} &
			kPa \\ \hline
			\multicolumn{1}{|c|}{\textbf{Viscosidade dinâmica}} &
			\multicolumn{4}{c|}{1.5673x10-6} &
			kPa   s \\ \hline
		\end{tabular}
		\caption{Propriedades utilizadas para modelar o solo coesivo não drenado}
		\label{tab:propriedades_crav}
	\end{table}
	
	No modelo, o revestimento condutor possui propriedades do aço é considerado como um corpo rígido, com densidade de 7860 kg/m³.   

	\subsection{Análise da influência do ângulo de dilatância}

Com base nas propriedades definidas para o solo e para o condutor, foram realizadas análises numéricas destinadas a investigar a influência dos parâmetros constitutivos na resposta do sistema. Entre eles, destaca-se o ângulo de dilatância, cuja variação foi utilizada para avaliar seus efeitos no comportamento de penetração e na resistência mobilizada durante a cravação.

O ângulo de dilatância $(\psi)$ foi conceituado como uma medida da expansão volumétrica durante o cisalhamento em materiais granulares. Sua relação com a resistência ao cisalhamento foi formalizada por \citet{rowe1962stress} através da teoria da dilatância por intertravamento de partículas, expressa pela Equação \ref{eq:dilatancia}: 
	
	\begin{equation}
	\frac{\sigma1}{\sigma3}=\left( \frac{1+sin\phi}{1-sin\phi} \right)\bullet \left( \frac{1+sin\psi}{1+sin\psi} \right)
	\label{eq:dilatancia}
	\end{equation}

O ângulo de dilatância, em análises geotécnicas, costuma ser assumido
como constante ao longo do processo de plastificação. O valor $\psi = 0$ indica que
o volume do solo é preservado durante o cisalhamento. Em solos argilosos,
mesmo na presença de sobreconsolidação, a dilatância geralmente é muito
baixa ($\psi$$\sim$ 0). Já em materiais granulares, como areias e cascalhos, com um
ângulo de atrito interno $\varphi>$ 30°, o ângulo de dilatância apresenta forte
dependência do atrito interno. Para solos não coesivos, é comum estimar o
ângulo de dilatância como mostra a Equação \ref{eq:dilatancia2}:

	\begin{equation}
	\psi=\varphi-30°
	\label{eq:dilatancia2}
\end{equation}

Neste estudo a respeito da influência do ângulo de dilatância, foi realizada uma análise numérica do processo de cravação de um revestimento condutor em solo não coesivo, representado como um material granular modelado pelo critério constitutivo de Mohr-Coulomb. O domínio de solo foi definido com propriedades geotécnicas típicas de materiais arenosos, incluindo ângulo de atrito interno, módulo de elasticidade e coesão.

Com o objetivo de avaliar a influência do ângulo de dilatância (($\psi$) no comportamento do solo durante o processo de cravação, foram consideradas diferentes variações desse parâmetro, adotando-se os valores de 0°, 5°, 10°, 15° e 20°. Essa estratégia permitiu analisar a sensibilidade da resposta numérica às distintas intensidades de dilatação, possibilitando identificar seus efeitos tanto no comportamento do solo quanto na interação com o revestimento ao longo da simulação.

A análise dos resultados concentrou-se na influência do ângulo de dilatância na profundidade de penetração do condutor ao longo do tempo(Figura \ref{fig:Figura_dilatancia01}). Observa-se que todas as simulações atingem uma profundidade máxima, porém com taxas de penetração distintas. De modo geral, o aumento do ângulo de dilatância resulta na redução da profundidade alcançada, indicando maior resistência do solo à cravação, como pode ser observado na Figura \ref{fig:Figura_dilatancia01}.

	\begin{figure}[H]
	\centering
	\includegraphics[width=0.7\linewidth]{cravacao/Figura_dilatancia01.pdf}
	\caption{Gráfico comparativo de profundidade atingida com os diferentes ângulos de dilatância (Autores, 2025)}
	\label{fig:Figura_dilatancia01}
\end{figure}

A Figura \ref{fig:Figura_dilatancia02} mostra a distribuição das tensões efetivas no solo durante a cravação do condutor. Observa-se concentração de tensões próximas à ponta e ao entorno do revestimento, indicando as regiões onde a resistência do solo é mobilizada. Uma zona intermediária com gradiente de tensões sugere plastificação e rearranjo das partículas, enquanto áreas mais distantes apresentam menores níveis de tensão por ainda não terem sido significativamente solicitadas. Esses resultados evidenciam a redistribuição de tensões causada pela penetração e reforçam a influência do comportamento dilatante na resposta do maciço.

	\begin{figure}[H]
	\centering
	\includegraphics[width=0.6\linewidth]{cravacao/Figura_dilatancia02.pdf}
	\caption{Magnitude da tensão efetiva ao longo do tempo (Autores, 2025)}
	\label{fig:Figura_dilatancia02}
\end{figure}

Conclui-se que o ângulo de dilatância exerce influência significativa na resposta do solo durante a cravação do revestimento, afetando a profundidade de penetração e distribuição de tensões. Os resultados evidenciam a sensibilidade do modelo numérico a esse parâmetro e reforçam a importância de sua adequada consideração para representar de forma realista a interação solo-estrutura em solos não coesivos.

\subsection{MPM Puro e MPM de integração mista}

O MPM consiste em uma abordagem híbrida Euleriana-Lagrangiana que combina as vantagens de métodos baseados em partículas e em malha. Nesse contexto, o domínio computacional é discretizado por uma malha Euleriana de fundo, enquanto os corpos materiais são representados por partículas Lagrangianas que transportam todas as propriedades constitutivas, como tensão, deformação e variáveis de histórico, movimentando-se através da malha. Essa dupla representação permite tratar grandes deformações de forma natural, evitando problemas de distorção de malha típicos de formulações puramente Lagrangianas, ao mesmo tempo em que preserva o acompanhamento do histórico do material.

A simulação da penetração de um corpo rígido em solo envolve deformações elevadas, situação em que o esquema convencional do MPM pode apresentar instabilidades numéricas associadas ao cruzamento de partículas entre elementos. Quando uma partícula atravessa a fronteira de um elemento, descontinuidades nos gradientes das funções de forma podem gerar forças nodais não físicas, reduzindo a precisão da solução.

Como alternativa para mitigar esse efeito, emprega-se o esquema de integração mista implementado no Anura3D. Nesse procedimento, calcula-se inicialmente uma tensão representativa constante para cada elemento, obtida pela média ponderada das tensões dos pontos materiais nele contidos. Em seguida, as forças internas são avaliadas por meio de integração de Gauss, de forma análoga aos procedimentos clássicos do Método dos Elementos Finitos, proporcionando maior estabilidade e suavidade na distribuição de tensões.

Considerando esse contexto, buscou-se analisar a operação da instalação de revestimento condutor em solos coesivos sob condição não drenada, típicos de ambientes offshore, por meio da comparação entre duas abordagens numéricas: o MPM convencional e o esquema de integração MPM-Mixed, que incorpora integração de Gauss. Assim, teve como objetivo avaliar qual formulação apresentava melhor desempenho e adaptabilidade para a simulação do processo.

A Figura \ref{fig:Figura_mpm01} apresenta a comparação entre os esquemas de integração entre o MPM puro e o MPM-Mixed. Observa-se que o MPM puro gera distribuições de tensões mais irregulares, com flutuações abruptas e zonas localizadas de alta intensidade, indicando a presença de instabilidades numéricas associadas ao cruzamento de células. Em contraste, o esquema MPM-Mixed produz campos de tensão mais suaves e contínuos, com transições progressivas entre regiões de diferentes magnitudes e redução de picos espúrios próximos às fronteiras rígidas. 

	\begin{figure}[H]
	\centering
	\includegraphics[width=0.8\linewidth]{cravacao/Figura_mpm01.pdf}
	\caption{Comparação da magnitude da tensão efetiva (kPa) para: (a) MPM puro; (b) Integração mista de MPM; (c) Profundidade de penetração versus tempo (Autores, 2025).}
	\label{fig:Figura_mpm01}
\end{figure}

Além disso, a análise da evolução da profundidade ao longo do tempo mostra que ambos os métodos atingem a profundidade final semelhante, porém a integração mista apresenta maior eficiência computacional, alcançando o resultado em menor tempo. Esses resultados evidenciam a superior estabilidade e desempenho do esquema MPM-Mixed para a simulação do problema analisado.

	\subsection{Resultados atuais}
	
	Os resultados do modelo numérico desenvolvido mostram uma forte concordância com os dados operacionais reais, validando a aplicação do Método do Ponto Material (MPM) na simulação da instalação de revestimentos de condutores em solos argilosos offshore. A precisão na previsão dos deslocamentos e das distribuições de tensão ao redor do condutor confirma a adequação das formulações empregadas para representar as complexas interações solo-estrutura em ambientes offshore desafiadores.
	
	A análise das distribuições de tensão ao longo das fases de peso próprio e martelamento indicou um aumento significativo da resistência do solo com a profundidade, o que contribui para a estabilização da taxa de penetração do revestimento. Esse comportamento está em conformidade com estudos prévios (\citet{galavi2024mpm}; \citet{martinelli2021investigation}), que também relataram um incremento progressivo da resistência do solo durante a instalação de estruturas similares.
	
	A simulação forneceu insights valiosos sobre o comportamento do solo durante a instalação do revestimento do condutor. A estabilização progressiva da descida durante a fase de peso próprio indica que o modelo representa com precisão a interação dinâmica entre o condutor e o solo circundante. Além disso, a resposta do solo ao início dos impactos de martelamento e o consequente aumento da resistência evidenciam a capacidade do modelo de capturar adequadamente as condições de carga dinâmica. Para uma melhor compreensão dos efeitos desse processo, a Figura \ref{fig:cravacao_result1} apresenta a distribuição de tensão no solo em dois momentos distintos: (a) após a fase de peso próprio e (b) após a fase de martelamento.
	
	\begin{figure}[H]
		\centering
		\includegraphics[width=0.7\linewidth]{cravacao/cravacao_result1.pdf}
		\caption{a) Distribuição da tensão vertical efetiva no final da fase de penetração do peso próprio; b) Distribuição da tensão vertical efetiva no final da fase de penetração do martelo. Fonte: Autores (2025)}
		\label{fig:cravacao_result1}
	\end{figure}
	
	A Figura \ref{fig:cravacao_result1} apresenta a distribuição das tensões verticais efetivas no solo. Uma análise mais detalhada das tensões próximas ao revestimento do condutor revela um aumento significativo na concentração de tensões na região da ponta. Esse comportamento é observado tanto na fase de peso próprio (Fig. \ref{fig:cravacao_result1}a) quanto após a aplicação das cargas de impacto (Fig. \ref{fig:cravacao_result1}b).
	
	Além disso, o perfil de descida do condutor ao longo do tempo pode ser analisado na Figura \ref{fig:cravacao_result2}. No início da operação, durante a fase de peso próprio, a velocidade de descida do condutor reduz gradualmente até se estabilizar por volta dos 5 segundos. A partir desse ponto, iniciam-se os impactos de martelamento.
	
	\begin{figure}[H]
		\centering
		\includegraphics[width=0.8\linewidth]{cravacao/cravacao_result2.pdf}
		\caption{Profundidade alcançada após a fase de martelamento. Atualmente, o condutor está em uma profundidade de aproximadamente 13,50 metros. Fonte: Autores (2025)}
		\label{fig:cravacao_result2}
	\end{figure}
	
	Durante a fase de peso próprio, o revestimento do condutor atinge aproximadamente 9,5 metros de profundidade após 5 segundos, penetrando a primeira camada de solo e adentrando a segunda, que apresenta maior resistência. Normalmente, essa fase resulta em profundidades entre 10 e 15 metros, indicando que o valor observado está próximo do esperado.
	
	Em seguida, a aplicação dos impactos provoca uma descida gradual do condutor. Com base na metodologia de \citet{galavi2019}, adotou-se um intervalo de 1 segundo entre impactos. Após um total de 64 impactos, o condutor alcançou cerca de 13,5 metros de profundidade. O estágio de martelamento é um processo complexo, pois a crescente resistência do solo exige um número progressivamente maior de impactos para cada avanço adicional. Para garantir a precisão do modelo na previsão da interação solo-estrutura, foi realizada uma calibração com dados operacionais da indústria, conforme ilustrado na Figura \ref{fig:cravacao_result3}.
	
	\begin{figure}[H]
		\centering
		\includegraphics[width=0.7\linewidth]{cravacao/cravacao_result3.pdf}
		\caption{Número acumulado de impactos ao longo da profundidade alcançada somente durante a fase de penetração do martelo. Fonte: Autores (2025)}
		\label{fig:cravacao_result3}
	\end{figure}
	
	A Figura \ref{fig:cravacao_result3} compara os dados operacionais da indústria (curva azul) com os resultados simulados (curva vermelha), evidenciando um comportamento semelhante entre ambos. Ajustes futuros no modelo e refinamentos nas simplificações aplicadas podem melhorar ainda mais essa correspondência.
	
	Assim, apesar dos avanços, algumas limitações devem ser consideradas. O modelo 2D axisimétrico, embora reduza custos computacionais, pode não representar completamente os efeitos tridimensionais das interações solo-estrutura, especialmente em solos anisotrópicos ou com variações laterais. A ampliação para um modelo 3D pode fornecer uma análise mais detalhada do processo de instalação.
	
	Além disso, a caracterização do solo baseada em dados CPTu pode apresentar variações conforme o local de instalação, impactando a precisão do modelo. Dessa forma, sua aplicação em diferentes contextos geotécnicos depende diretamente da qualidade e representatividade dos dados de entrada.
	
	\subsection{Considerações finais}
	
	O estudo da cravação do revestimento condutor tem avançado significativamente nas etapas mais recentes, demonstrando que o modelo numérico desenvolvido está alinhado com o problema proposto, desempenhando um papel fundamental na previsão do comportamento do condutor durante a instalação. Nesse contexto, a análise evidenciou a influência das propriedades do solo e das condições de cravação na profundidade final atingida, permitindo ajustes mais precisos nos parâmetros operacionais. No entanto, a modelagem adotada foi limitada a um formato axissimétrico, o que pode restringir a representação completa dos efeitos tridimensionais da interação solo-estrutura, além de outras simplificações relacionadas à etapa de martealmento. Diante disso, estudos futuros devem explorar a ampliação do modelo, visando maior precisão nas simulações e uma compreensão mais detalhada dos fatores que influenciam o processo de cravação.