% Montagem do acompanhamento do projeto, com apresentação das metas e atividades realizadas

\chapter{Acompanhamento das Atividades}

\setlength{\headheight}{50pt}

O projeto foi dividido em macroetapas para facilitar o acompanhamento das atividades planejadas e executadas. São elas:

\begin{enumerate}
	\item Planejamento;

	\item Desenvolvimentos relacionados à avaliação de parâmetros de capacidade de carga em solos;

	\item Desenvolvimentos relacionados ao projeto de sistema de revestimento conduto;

	\item Estudos numéricos para avaliação integrada de sistemas de início de poço;

	\item Desenvolvimentos relacionados a geotécnica de poços em rochas salinas;

	\item Controle;

	\item Encerramento;

%	\item Desenvolvimento de análises da interação solo-revestimento em Método dos Elementos Finitos;

%	\item Acompanhamento do projeto.

%	\item Desenvolvimento de modelos para aprimoramento de avaliação de resistência do solo

%	\item Avaliação de integridade de revestimento condutor/superfície em cenários críticos de operação


%	\item Estudos numéricos para otimização de parâmetros de instalação de condutor

\end{enumerate}

Nas Tabelas \ref{tab:cronograma1} e \ref{tab:cronogram2} podem-se observar as macroetapas e as atividades com execução iniciada ou em fase de preparação, com seus respectivos prazos e percentual de execução. %Vale ressaltar que o projeto possui data de início no 31/07/2018, mas a contratação da equipe teve início no 01/10/2018, em consonância com a data do primeiro desembolso financeiro.

\begin{table}[H]
	\centering
	\begin{tabularx}{\hsize}{|l|X|l|l|l|l|}\hline
		\hline
		\multicolumn{1}{|c|}{Etapas} & \multicolumn{1}{c|}{Nome} & \multicolumn{1}{c|}{Mês Inicial} & \multicolumn{1}{c|}{Mês Final} & \multicolumn{1}{c|}{Duração} & \% Executado \\ \hline
		1 & Atividade P1 - Planejamento das atividades e mobilização da equipe de trabalho & 1 & 1 & 1 & 100 \\ \hline
		2 & S1.I Geração automática de folha executiva de início de poço & 1 & 12 & 12 & 60 \\ \hline
		2 & S1.V Atualização do modelo de cravabilidade em Base Torpedo & 1 & 12 & 12 & 75 \\ \hline
		2 & S1.VII Técnicas de aprendizado de máquina na modelagem de solos & 1 & 36 & 36 & 10 \\ \hline
		2 & S1.VIII Atualização dos recursos e tecnologias computacionais utilizados no sistema SEST SOLOS & 1 & 36 & 36 & 15 \\ \hline
		2 & S1.IV capacidade de carga do solo com presença de zonas de intercalação de areia/argila & 7 & 18 & 12 & 10 \\ \hline
		2 & S1.VI  Modelagem de solos considerando incertezas dos ensaios & 7 & 18 & 12 & 30 \\ \hline
		2 & S1.II Desenvolvimentos para cálculo de capacidade de carga em solos considerando efeitos térmicos & 13 & 24 & 12 & 10 \\ \hline
		3 & S2.I  Análise estrutural e de estabilidade geotécnica de condutores cravados & 1 & 24 & 24 & 50 \\ \hline
		3 & S2.II Aprimoramento de modelo em elementos finitos para interação sistema solo-condutor & 1 & 24 & 24 & 20 \\ \hline
		3 & S2.VI Atualização dos recursos e tecnologias computacionais utilizados no sistema SIMCON & 1 & 36 & 36 & 30 \\ \hline
		3 & S2.IV Aprimoramento de modelo mecano-fiabilístico para integridade solo-revestimento & 25 & 36 & 12 & 15 \\ \hline
		3 & S2.V Metodologia para geração automática de relatório de projeto de início de poço & 25 & 36 & 12 & 10 \\ \hline
		4 & S3.I Estudos numéricos para otimização de instalação de condutor & 1 & 36 & 36 & 25 \\ \hline
		4 & S3.II Estudos numéricos para otimização de instalação de condutor & 25 & 36 & 36 & 5 \\ \hline
		\end{tabularx}
	\caption{Cronograma físico do projeto. Fonte: Autores (2024)}
	\label{tab:cronograma1}
\end{table}

\begin{table}[H]
			\centering
		\begin{tabularx}{\hsize}{|l|X|l|l|l|l|}\hline
				\hline
				\multicolumn{1}{|c|}{Etapas} & \multicolumn{1}{c|}{Nome} & \multicolumn{1}{c|}{Mês Inicial} & \multicolumn{1}{c|}{Mês Final} & \multicolumn{1}{c|}{Duração} & \% Executado \\ \hline
		5 & S4.I Aprimoramento de simulador de fluência salina & 1 & 12 & 12 & 100 \\ \hline
		5 & S4.II Incorporação de lei de fluência primária e terciária no simulador SEST SAL & 1 & 30 & 30 & 40 \\ \hline
		5 & S4.VII Modelagem do Leakoff Test em regiões com rochas salinas & 1 & 30 & 30 & 50 \\ \hline
		5 & S4.X Atualização dos recursos e tecnologias computacionais utilizados no sistema SEST SAL & 1 & 36 & 36 & 50 \\ \hline
		5 & S4.III Redefinição de elementos devido ao repasse de broca em rocha salina em modelos 2D & 13 & 24 & 12 & 90 \\ \hline
		5 & S4.IX Calibração de modelo de fluência salina com retroanálise de prisão de coluna de perfuração 2D & 13 & 36 & 24 & 0 \\ \hline
		5 & S4.VI Manutenção e melhorias no módulo integrado ao PWDA - SEST SAL & 13 & 36 & 24 & 0 \\ \hline
		5 & S4.IV Módulo computacional para recomendação de repasses e do peso do fluido de perfuração & 25 & 36 & 12 & 0 \\ \hline
		5 & S4.V  Elaboração de relatório automático no sistema SEST SAL & 25 & 36 & 12 & 0 \\ \hline
		5 & S4.VIII  Incorporação do simulador computacional LOT ao POÇO WEB & 25 & 36 & 12 & 0 \\ \hline
		6 & Atividade R1 - Reunião 01 de acompanhamento & 10 & 10 & 1 & 0 \\ \hline
		6 & Atividade R2 - Reunião 02 de acompanhamento & 22 & 22 & 1 & 0 \\ \hline
		6 & Atividade R3 - Reunião 02 de acompanhamento & 22 & 22 & 1 & 0 \\ \hline
		7 & Atividade RF -Encerramento do instrumento contratual  & 36 & 36 & 1 & 0 \\ \hline
	\end{tabularx}
	\caption{Cronograma físico do projeto. Fonte: Autores (2024)}
	\label{tab:cronogram2}
\end{table}

%\begin{table}[H]
%	\centering
%	\begin{tabularx}{\hsize}{|l|X|l|l|l|l|}\hline
%		\hline
%		\multicolumn{1}{|c|}{Etapas} & \multicolumn{1}{c|}{Nome} & \multicolumn{1}{c|}{Mês Inicial} & \multicolumn{1}{c|}{Mês Final} & \multicolumn{1}{c|}{Duração} & \% Executado \\ \hline
		%3 & A3.1 Revisão bibliográfica sobre jateamento de sistemas de revestimento condutor & 6 & 7 & 2 & 100 \\ \hline
		%3 & A3.2 Modelagem numérica de operações de jateamento & 10 & 23 & 14 & 100 \\ \hline
		%3 & A3.3 Calibração do modelo numérico de operações de jateamento & 24 & 33 & 10 & 100 \\ \hline
%		&  &  &  &  &  \\ \hline
		%4 & A4.1 Revisão Bibliográfica sobre Cravação por martelamento & 6 & 7 & 2 & 100 \\ \hline
		%4 & A4.2 Modelagem de Operações de Cravação por martelamento & 10 & 23 & 14 & 100 \\ \hline
		%4 & A4.3 Calibração do Modelo Numérico de Operações de Cravação por martelamento & 24 & 34 & 11 & 100 \\ \hline
%		&  &  &  &  &  \\ \hline
		%5 & A5 Desenvolvimentos de análises da interação solo revestimento em método dos elementos finitos & 10 & 33 & 24 & 100 \\ \hline
%		&  &  &  &  &  \\ \hline
		%6 & A6 Acompanhamento do projeto & 1 & 36 & 36 & 100 \\
%		\hline
%		&  &  &  &  &  \\ \hline
		%7 & A7 Desenvolvimento de modelos para aprimoramento de avaliação de resistência do solo & 36 & 42 & 6 & 100 \\
		%\hline
		%7 & A7.1 Estudos e desenvolvimentos de modelos de resistência do solo com base em série de dados & 38 & 49 & 12 & 100 \\
%		\hline
		%7 & A7.2 A7.2 Aprimoramento do módulo de cravabilidade de Base Torpedo & 47 & 58 & 12 & 100 \\
%		\hline
%		&  &  &  &  &  \\ \hline
		%8 & A8.1 Atividades para avaliação de cenários críticos de operação no condutor/revestimento & 38 & 49 & 12 & 100 \\
%		\hline
		%8 &A8.2 Implementação do condutor cravado e inclusão de novos modelos de interação solo-estrutura & 47 & 58 & 12 & 100 \\
%		\hline
		%8 &A8.3 Aprimoramentos na modelagem de revestimento condutor/superfície & 38 & 57 & 20 & 100 \\
%		\hline
		%8 & A8.4 Dimensionamento probabilístico das fundações e análise paramétrica para otimização & 38 & 49 & 12 & 100 \\
%		\hline
		%8 & A8.5 Módulo para cálculo de projetos em batelada e integração com Simwear & 47 & 58 & 12 & 100 \\
%		\hline
%		&  &  &  &  &  \\ \hline
%		9 & A9.1 Estudos numéricos para otimização de parâmetros de instalação de condutor & 38 & 58 & 21 & 100 \\
%	\hline
%	\end{tabularx}
%\caption{Cronograma físico do projeto. Fonte: Autores (2021)}
%\label{tab:cronograma}
%\end{table}