% !TeX encoding = UTF-8

\pdfminorversion=7

\documentclass{lccvreport}

% Arquivo com os pacotes a serem utilizados
%%%%%%%%%%%%%%%%%%%%%%
%%% PACOTES EM USO
%
%      geometry
%      inputenc
%      fontenc
%      babel
%      fancyhdr
%      lastpage
%      natbib
%      graphicx



%%%%%%%%%%%%%%%%%%%%%%
%%% INSIRA SEUS PACOTES AQUI

% NECESSARIOS PARA RODAR
\usepackage{tabularx}
\usepackage{multirow}
\usepackage{multicol}
\usepackage{float}
\usepackage[table,xcdraw]{xcolor}
\usepackage{csquotes}
\usepackage{amsmath}
\usepackage{caption}
\usepackage{subcaption}
\usepackage{xltabular}

% \usepackage{hyperref}
% \usepackage{dirtree}
% \usepackage{indentfirst}
% \usepackage{amsfonts}
% \usepackage{adjustbox}
% \usepackage{booktabs}
% \usepackage{longtable}
% \usepackage{textcomp}
% \usepackage{enumitem}
% \usepackage{siunitx}


%% Configurações adicionais aos pacotes
\setlength{\parindent}{1cm}  % Inicio do paragrafo

% Arquivo com as informações da capa do projeto
%%%%%%%%%%%%%%%%%%%%%%
%%% INFORMAÇÕES DA CAPA

%Colocar aqui o tipo de relatório
\type{Relatório Técnico Parcial 01}

%Colocar aqui o título do projeto
\title{Desenvolvimento de métodos e aplicativos computacionais de geomecânica salina e geotécnica de poço}

%Colocar aqui o apelido do projeto
\shorttitle{Projeto SOLOS} % substituior o nome APELIDO

%Colocar aqui o nome do coordenador do projeto.
%Masculino:
\coordinator[Coordenador]{João Paulo Lima Santos}
%Feminino:
%\coordinator[Coordenadora]{Nome da Coordenadora}

%Aqui colocar o nome dos autores do documento (não precisa repetir o nome do coordenador(a))
\author{
 Aline da Silva Ramos Barboza \\
 Eduardo Toledo de Lima Junior \\
 Lucas Pereira de Gouveia \\
 Rodrigo de Barros Paes \\
 William Wagner Matos Lira \\
 Antonio Paulo Amancio Ferro \\
 Beatriz Ramos Barboza \\
 Catarina Nogueira de Araújo Fernandes \\
 Carlos Mikael Alencar Tenorio \\
 Christiano Augusto Ferrario Várady Filho \\
 Daniel de Melo Pimentel \\
 Davi Leão Ramos \\
 Diogo dos Santos Fonseca \\
 Elisama Quezia Silva Santos \\
 Emanuel Melo Silva \\
 Flávio Vasconcelos Pais \\
 Higor Vinícius de Lima \\
 Jamerson Braga de Omena \\
 Jennifer Mikaella Ferreira Melo \\
 Joab Manoel Almeida Santos 
 João Fyllipy de Lima Nunes \\
 João Victor Gonçalves Fernandes \\
 Joyce Kelly França Tenório \\
 Luis Philipe Ribeiro Almeida \\
 Mávyla Sandreya Correia Tenorio  \\
 Nuno Henrique Albuquerque Pires \\
 Tácio Valmir Dantas de Almeida \\
 Themisson dos Santos Vasconcelos \\
 Vitor Rafael Morais e Silva
}
%Colocar aqui o período do projeto (Informação retirada do SIGITEC)
\dateperiod{29/02/2024 a 03/03/2025} % Período a que se refere-se o relatório
%Colocar aqui o número do processo (Informação retirada do SIGITEC)
\processnumber{2023/00121-4} % Nº Processo - consultar SIGITEC
%Colocar aqui o número PT (Informação retirada do SIGITEC)
\ptnumber{PT-200.20.00048} % Nº PT - consultar SIGITEC
%Colocar aqui o SAP (Informação retirada do SIGITEC)
\sapnumber{4600674653} % Nº SAP - consultar SIGITEC
%Colocar aqui o número jurídico (Informação retirada do contrato/termo de convênio)
\legalnumber{5850.0108243.18.9} % Nº Jurídico - consultar CONTRATO/TERMO DE CONVÊNIO
%Colocar aqui a data de entrega do relatório
\date{31 de Maio de 2023} % Data de publicação (emissão inicial)


\addbibresource{bibliografia.bib}

\begin{document}

\maketitle

% Aquivo de controle de versão
%%Este capítulo é reservado para manter um controle das revisões do documento. Propõe-se que cada versão do mesmo relatório enviada para petrobrás seja uma revisão diferente do mesmo documento. Por exemplo: a versão inicial do `relatório parcial 01` seria a emissão inicial. Caso correções sejam solicitadas pelo contratante as próximas versões estariam documentadas aqui.

\chapter*{Controle de revisão}

No quadro abaixo é apresentado o controle de revisão do documento.

%para cada nova revisão uma nova linha na esturtura abaixo seria adicionada
\begin{revision}

\addrevision{Emissão inicial}{01/03/2024}{Beatriz Barboza}
\addrevision{Relatório 2025}{03/03/2025}{Beatriz Barboza}
%primeira revisão (comentar quando estiver trabalhando na emissão inicial)
\end{revision}


%%%%%%%%%%%%%%%   NÃO ALTERAR E NEM REMOVER   %%%%%%%%%%%%%%%%
\ifnum\value{revcounter}>-1\currentrevision{\therevcounter}\fi
%%%%%%%%%%%%%%%%%%%%%%%%%%%%%%%%%%%%%%%%%%%%%%%%%%%%%%%%%%%%%%

% NÃO MODIFICAR! Criação do sumário.
\tableofcontents\pdfbookmark[0]{Table of Contents}{toc}

% Arquivo com as informações do projeto, contratante, contratado e instituições financeiras envolvidas
%%%%%%%%%%%%%%%%%%%%%%
%%% DADOS GERAIS DO PROJETO

\setlength{\headheight}{50pt}

%Aqui colocar a data de início do projeto
\def\inicioprojeto{01 de julho de 2023}
%Aqui colocar a duração do projeto
\def\duracaoprojeto{36 meses}
\def\coordenador{\getcoordinator}
%Aqui colocar o CPF do coordenador(a) do projeto
\def\coordenadorCPF{039.083.094-17}
%Aqui colocar a função do coordenador(a) do projeto (por ex: professor titular, professor adjunto 1, etc...)
\def\funcao{Professor Adjunto}
%Aqui colocar o regime de contratação
\def\regime{Dedicação exclusiva}

%%%%%%%%%%%%%%%%%%%%%%%%%%%%%
%%%% 	DADOS DO LABORATÓRIO
\def\lccv{Laboratório de Computação Científica e Visualização - LCCV}

\def\ufal{Universidade Federal de Alagoas - UFAL}
\def\ufalCNPJ{24.464.109/0001-48}

%%%%%%%%%%%%%%%%%%%%%%%%%%%%%
%%%% DADOS DA FUNDAÇÃO - aqui colocou-se os dados da fundepes
%caso outra fundação seja utilizada, modificar de acordo
\def\fundepes{Fundação Universitária de Desenvolvimento de Extensão e Pesquisa - FUNDEPES}
\def\fundepesCNPJ{12.449.880/0001-67}

%%%%%%%%%%%%%%%%%%%%%%%%%%%%
%%%% DADOS DA EMPRESA CONTRATANTE DO PROJETO - aqui
% utilizou-se a petrobrás como exemplo, caso outra empresa seja a contratante modificar
% de acordo
\def\financiador{Petróleo Brasileiro S.A. - PETROBRAS}
\def\financiadorCNPJ{33.000.167/0001-01}


%%%%%%%%%%%%%%%%%%%%%%
%%% APRESENTAÇÃO

\chapter*{Apresentação}

O presente relatório refere-se ao projeto de pesquisa intitulado \gettitle, o qual teve sua contratação realizada em \inicioprojeto, com duração prevista de 36 meses. %Em 21/07/2021, o projeto foi aditivado para realização de atividades  complementares, perfazendo um prazo total de execução de \duracaoprojeto.
As informações gerais de identificação do projeto são apresentadas a seguir:


\begin{center}
\footnotesize
\begin{tabular}{| p{11.5cm} | p{3cm} |}
\hline
\multicolumn{2}{|c|}{\textbf{DADOS GERAIS DAS INSTITUIÇÕES ENVOLVIDAS}} \\
\hline
\textbf{Concedente:} & \textbf{CNPJ:}   \\
\financiador         & \financiadorCNPJ \\
\hline
\textbf{Convenente:} & \textbf{CNPJ:}   \\
\fundepes            & \fundepesCNPJ    \\
\hline
\textbf{Proponente:} & \textbf{CNPJ:}   \\
\ufal                & \ufalCNPJ        \\
\hline
\end{tabular}


\begin{tabular}{| p{7.25cm} | p{7.25cm} |}
\hline
\multicolumn{2}{|c|}{\textbf{DADOS GERAIS DO PROJETO}}      \\
\hline
\multicolumn{2}{|p{14.5cm}|}{\textbf{Título do projeto:}}   \\
\multicolumn{2}{|p{14.5cm}|}{\gettitle}                     \\
\hline
\textbf{Início do projeto:} & \textbf{Período de execução:} \\
\inicioprojeto              & \duracaoprojeto               \\
\hline
\textbf{Processo:}          & \textbf{No. PT:}              \\
\getprocessnumber           & \getptnumber                  \\
\hline
\textbf{No. SAP:}           & \textbf{No. Jurídico:}        \\
\getsapnumber               & \getlegalnumber               \\
\hline
\textbf{Coordenador:}       & \textbf{CPF:}                 \\
\coordenador                & \coordenadorCPF               \\
\hline
\textbf{Função:}            & \textbf{Regime de trabalho:}  \\
\funcao                     & \regime                       \\
\hline
\textbf{Instituição:}       & \textbf{Departamento:}        \\
\ufal                       & \lccv                         \\
\hline
\end{tabular}
\end{center}


% Resumo do projeto (informação retirada da proposta do projeto ou SIGITEC)
%%Resumo do projeto. Normalmente esse resumo já foi escrito durante o processo de elaboração do projeto. No caso da petrobras, esse texto pode ser retirado diretamente do SIGITEC.
\setlength{\headheight}{50pt}

\begin{abstract}

Este projeto visa o desenvolvimento de métodos e ferramentas computacionais de geomecânica salina e geotécnica de poços para avaliação de integridade na perfuração em rochas salinas e na gestão de integridade do sistema solo-revestimento em todo ciclo produtivo do poço. A metodologia de desenvolvimento deste projeto contempla as seguintes macroetapas: a) estudos e desenvolvimentos relacionados à avaliação de parâmetros de capacidade de carga em solos para projeto de início de poço (instalado por cravação, jateamento ou perfurado e cimentado); 2) estudos e desenvolvimentos relacionados ao projeto de sistema de revestimento condutor e superfície; 3) estudos numéricos para avaliação integrada solo-revestimento e otimização de parâmetros de dimensionamento; 4) estudos e desenvolvimentos relacionados a geomecânica de poços em rochas salinas e na realização de LOT. Este projeto contribui de forma significativa no desenvolvimento de operações de início de poço, perfuração em rochas salinas e na gestão de integridade ao longo da vida operacional, visando atendimento de normativos de segurança operacional.O presente relatório parcial 1 visa a apresentação das atividades técnicas realizadas no período de 01/07/2023 a 29/02/2024.

\end{abstract}

% Acompanhamento (Apresentação de metas e atividades executadas)
% Montagem do acompanhamento do projeto, com apresentação das metas e atividades realizadas

\chapter{Acompanhamento das Atividades}

\setlength{\headheight}{50pt}

O projeto foi dividido em macroetapas para facilitar o acompanhamento das atividades planejadas e executadas. São elas:

\begin{enumerate}
	\item Planejamento;

	\item Desenvolvimentos relacionados à avaliação de parâmetros de capacidade de carga em solos;

	\item Desenvolvimentos relacionados ao projeto de sistema de revestimento conduto;

	\item Estudos numéricos para avaliação integrada de sistemas de início de poço;

	\item Desenvolvimentos relacionados a geotécnica de poços em rochas salinas;

	\item Controle;

	\item Encerramento;

%	\item Desenvolvimento de análises da interação solo-revestimento em Método dos Elementos Finitos;

%	\item Acompanhamento do projeto.

%	\item Desenvolvimento de modelos para aprimoramento de avaliação de resistência do solo

%	\item Avaliação de integridade de revestimento condutor/superfície em cenários críticos de operação


%	\item Estudos numéricos para otimização de parâmetros de instalação de condutor

\end{enumerate}

Nas Tabelas \ref{tab:cronograma1} e \ref{tab:cronogram2} podem-se observar as macroetapas e as atividades com execução iniciada ou em fase de preparação, com seus respectivos prazos e percentual de execução. %Vale ressaltar que o projeto possui data de início no 31/07/2018, mas a contratação da equipe teve início no 01/10/2018, em consonância com a data do primeiro desembolso financeiro.

\begin{table}[H]
	\centering
	\begin{tabularx}{\hsize}{|l|X|l|l|l|l|}\hline
		\hline
		\multicolumn{1}{|c|}{Etapas} & \multicolumn{1}{c|}{Nome} & \multicolumn{1}{c|}{Mês Inicial} & \multicolumn{1}{c|}{Mês Final} & \multicolumn{1}{c|}{Duração} & \% Executado \\ \hline
		1 & Atividade P1 - Planejamento das atividades e mobilização da equipe de trabalho & 1 & 1 & 1 & 100 \\ \hline
		2 & S1.I Geração automática de folha executiva de início de poço & 1 & 12 & 12 & 60 \\ \hline
		2 & S1.V Atualização do modelo de cravabilidade em Base Torpedo & 1 & 12 & 12 & 75 \\ \hline
		2 & S1.VII Técnicas de aprendizado de máquina na modelagem de solos & 1 & 36 & 36 & 10 \\ \hline
		2 & S1.VIII Atualização dos recursos e tecnologias computacionais utilizados no sistema SEST SOLOS & 1 & 36 & 36 & 15 \\ \hline
		2 & S1.IV capacidade de carga do solo com presença de zonas de intercalação de areia/argila & 7 & 18 & 12 & 10 \\ \hline
		2 & S1.VI  Modelagem de solos considerando incertezas dos ensaios & 7 & 18 & 12 & 30 \\ \hline
		2 & S1.II Desenvolvimentos para cálculo de capacidade de carga em solos considerando efeitos térmicos & 13 & 24 & 12 & 10 \\ \hline
		3 & S2.I  Análise estrutural e de estabilidade geotécnica de condutores cravados & 1 & 24 & 24 & 50 \\ \hline
		3 & S2.II Aprimoramento de modelo em elementos finitos para interação sistema solo-condutor & 1 & 24 & 24 & 20 \\ \hline
		3 & S2.VI Atualização dos recursos e tecnologias computacionais utilizados no sistema SIMCON & 1 & 36 & 36 & 30 \\ \hline
		3 & S2.IV Aprimoramento de modelo mecano-fiabilístico para integridade solo-revestimento & 25 & 36 & 12 & 15 \\ \hline
		3 & S2.V Metodologia para geração automática de relatório de projeto de início de poço & 25 & 36 & 12 & 10 \\ \hline
		4 & S3.I Estudos numéricos para otimização de instalação de condutor & 1 & 36 & 36 & 25 \\ \hline
		4 & S3.II Estudos numéricos para otimização de instalação de condutor & 25 & 36 & 36 & 5 \\ \hline
		\end{tabularx}
	\caption{Cronograma físico do projeto. Fonte: Autores (2024)}
	\label{tab:cronograma1}
\end{table}

\begin{table}[H]
			\centering
		\begin{tabularx}{\hsize}{|l|X|l|l|l|l|}\hline
				\hline
				\multicolumn{1}{|c|}{Etapas} & \multicolumn{1}{c|}{Nome} & \multicolumn{1}{c|}{Mês Inicial} & \multicolumn{1}{c|}{Mês Final} & \multicolumn{1}{c|}{Duração} & \% Executado \\ \hline
		5 & S4.I Aprimoramento de simulador de fluência salina & 1 & 12 & 12 & 100 \\ \hline
		5 & S4.II Incorporação de lei de fluência primária e terciária no simulador SEST SAL & 1 & 30 & 30 & 40 \\ \hline
		5 & S4.VII Modelagem do Leakoff Test em regiões com rochas salinas & 1 & 30 & 30 & 50 \\ \hline
		5 & S4.X Atualização dos recursos e tecnologias computacionais utilizados no sistema SEST SAL & 1 & 36 & 36 & 50 \\ \hline
		5 & S4.III Redefinição de elementos devido ao repasse de broca em rocha salina em modelos 2D & 13 & 24 & 12 & 90 \\ \hline
		5 & S4.IX Calibração de modelo de fluência salina com retroanálise de prisão de coluna de perfuração 2D & 13 & 36 & 24 & 0 \\ \hline
		5 & S4.VI Manutenção e melhorias no módulo integrado ao PWDA - SEST SAL & 13 & 36 & 24 & 0 \\ \hline
		5 & S4.IV Módulo computacional para recomendação de repasses e do peso do fluido de perfuração & 25 & 36 & 12 & 0 \\ \hline
		5 & S4.V  Elaboração de relatório automático no sistema SEST SAL & 25 & 36 & 12 & 0 \\ \hline
		5 & S4.VIII  Incorporação do simulador computacional LOT ao POÇO WEB & 25 & 36 & 12 & 0 \\ \hline
		6 & Atividade R1 - Reunião 01 de acompanhamento & 10 & 10 & 1 & 0 \\ \hline
		6 & Atividade R2 - Reunião 02 de acompanhamento & 22 & 22 & 1 & 0 \\ \hline
		6 & Atividade R3 - Reunião 02 de acompanhamento & 22 & 22 & 1 & 0 \\ \hline
		7 & Atividade RF -Encerramento do instrumento contratual  & 36 & 36 & 1 & 0 \\ \hline
	\end{tabularx}
	\caption{Cronograma físico do projeto. Fonte: Autores (2024)}
	\label{tab:cronogram2}
\end{table}

%\begin{table}[H]
%	\centering
%	\begin{tabularx}{\hsize}{|l|X|l|l|l|l|}\hline
%		\hline
%		\multicolumn{1}{|c|}{Etapas} & \multicolumn{1}{c|}{Nome} & \multicolumn{1}{c|}{Mês Inicial} & \multicolumn{1}{c|}{Mês Final} & \multicolumn{1}{c|}{Duração} & \% Executado \\ \hline
		%3 & A3.1 Revisão bibliográfica sobre jateamento de sistemas de revestimento condutor & 6 & 7 & 2 & 100 \\ \hline
		%3 & A3.2 Modelagem numérica de operações de jateamento & 10 & 23 & 14 & 100 \\ \hline
		%3 & A3.3 Calibração do modelo numérico de operações de jateamento & 24 & 33 & 10 & 100 \\ \hline
%		&  &  &  &  &  \\ \hline
		%4 & A4.1 Revisão Bibliográfica sobre Cravação por martelamento & 6 & 7 & 2 & 100 \\ \hline
		%4 & A4.2 Modelagem de Operações de Cravação por martelamento & 10 & 23 & 14 & 100 \\ \hline
		%4 & A4.3 Calibração do Modelo Numérico de Operações de Cravação por martelamento & 24 & 34 & 11 & 100 \\ \hline
%		&  &  &  &  &  \\ \hline
		%5 & A5 Desenvolvimentos de análises da interação solo revestimento em método dos elementos finitos & 10 & 33 & 24 & 100 \\ \hline
%		&  &  &  &  &  \\ \hline
		%6 & A6 Acompanhamento do projeto & 1 & 36 & 36 & 100 \\
%		\hline
%		&  &  &  &  &  \\ \hline
		%7 & A7 Desenvolvimento de modelos para aprimoramento de avaliação de resistência do solo & 36 & 42 & 6 & 100 \\
		%\hline
		%7 & A7.1 Estudos e desenvolvimentos de modelos de resistência do solo com base em série de dados & 38 & 49 & 12 & 100 \\
%		\hline
		%7 & A7.2 A7.2 Aprimoramento do módulo de cravabilidade de Base Torpedo & 47 & 58 & 12 & 100 \\
%		\hline
%		&  &  &  &  &  \\ \hline
		%8 & A8.1 Atividades para avaliação de cenários críticos de operação no condutor/revestimento & 38 & 49 & 12 & 100 \\
%		\hline
		%8 &A8.2 Implementação do condutor cravado e inclusão de novos modelos de interação solo-estrutura & 47 & 58 & 12 & 100 \\
%		\hline
		%8 &A8.3 Aprimoramentos na modelagem de revestimento condutor/superfície & 38 & 57 & 20 & 100 \\
%		\hline
		%8 & A8.4 Dimensionamento probabilístico das fundações e análise paramétrica para otimização & 38 & 49 & 12 & 100 \\
%		\hline
		%8 & A8.5 Módulo para cálculo de projetos em batelada e integração com Simwear & 47 & 58 & 12 & 100 \\
%		\hline
%		&  &  &  &  &  \\ \hline
%		9 & A9.1 Estudos numéricos para otimização de parâmetros de instalação de condutor & 38 & 58 & 21 & 100 \\
%	\hline
%	\end{tabularx}
%\caption{Cronograma físico do projeto. Fonte: Autores (2021)}
%\label{tab:cronograma}
%\end{table}

% Introdução (Informações normalmente retiradas da proposta do projeto ou SIGITEC)
%%Toda a introdução pode ser retirada da proposta do projeto (o pdf do SIGITEC). Nesse pdf todas as seções abaixo estão detalhadas, basta copiar e colar aqui, mantendo as citações, se houverem.
\chapter{Introdução}

\setlength{\headheight}{50pt}

No projeto de P,D\&I intitulado "Técnicas de modelagem numéricas aplicadas a estimativa de propriedades do solo para projetos de poços de petróleo'' (SAP 4600564610), executado em parceria pelo Centro de Pesquisas da PETROBRAS (CENPES/PETROBRAS) e o Laboratório de Computação Científica e Visualização da Universidade Federal de Alagoas (LCCV/UFAL), foram desenvolvidos estudos e implementações para apoio ao dimensionamento de sistemas de projeto de início de poço. Em outros desenvolvimentos do LCCV/UFAL, foram construídas metodologias e implementadas ferramentas para avaliação de perfuração em rochas salinas. No presente projeto, serão desenvolvidas novas metodologias para apoio a projetos na fase de início de poço e na perfuração de rochas salinas, sendo incorporadas nos sistemas computacionais denominados SEST SAL, SEST SOLOS e SIMCON, hospedados no ambiente colaborativo POÇOWEB, o qual congrega diversas ferramentas utilizadas em projeto de poços na PETROBRAS.

Segundo Lacasse et al. (2007), os solos naturalmente apresentam propriedades variáveis devido ao próprio processo de formação, o que acarreta incertezas em suas características mecânicas. Para avaliação de técnicas de projeto de início de poço em regiões com cenários geológicos complexos envolvendo diferentes intercalações de argila e areia, a atividade de "cálculo incremental da capacidade de carga ao longo da profundidade" mostra-se importante para garantia de integridade de sistemas de revestimento condutor-superfície. Além disso, atualizações na "modelagem de solos considerando incertezas dos ensaios" e desenvolvimento de "Estudos para aplicação de técnicas de aprendizado de máquina na modelagem de solos" tornam-se elementos importantes de projeto de revestimento condutor-superfície, pois auxiliam no processo de tomada de decisão e quantificação de incertezas, além de seguir as diretrizes de metodologias consagradas na indústria, a exemplo da DNV-RP-C207 (2012).

A atualização de metodologias para caracterização do perfil de resistência mecânica de solos visando projeto de início de poço é importante para tornar mais fidedignas as estimativas de projeto. Nesse sentido, o cálculo de capacidade de carga de solo com consideração de adensamento térmico torna mais confiável as estimativas no sistema SOLOS para poços injetores ao considerar no cálculo eos gradientes térmicos envolvidos na fase de operação. Da mesma forma, a incorporação da influência na capacidade de carga do solo de efeitos de vibração do motor no caso de instalação de condutor jateado mostra-se como uma atividade importante frente as crescentes demandas de operações de jateamento na indústria.

A avaliação de integridade do revestimento condutor deve ser realizada a partir de uma análise acoplada entre a capacidade de carga do solo (mensurada a partir da estimativa da resistência ao cisalhamento não drenado do solo) e a integridade mecânica do sistema de revestimento do poço. O revestimento condutor deve resistir aos esforços conjuntos provocados pelo peso do próprio sistema e cargas na fase de operação, bem como pelos esforços adicionais transferidos pelo riser na fase de perfuração. Nesse contexto, o presente projeto propõe o desenvolvimento de métodos e ferramentas para "análise estrutural e de estabilidade geotécnica de condutores cravados". Também estão previstas atualizações do elemento finito usado no SIMCON para avaliar rotação na cabeça de poço por meio da integração de carregamentos laterais, que visa o estudo e desenvolvimento de metodologia para avaliação da interação solo-revestimento, com base nas reações laterais e axiais do sistema solo-revestimento. Também estão propostos no corrente projeto o aprimoramento de modelo mecano-fiabilístico com base em dados do solo e cenários de carregamento, de maneira a trazer ao projetista informações complementares de probabilidade de falha da estrutura. No sistema SIMCON, também serão desenvolvidas atividades visando o aprimoramento de avaliação de desgaste em condutor com integração completa com o sistema de avaliação de desgaste em revestimento desenvolvido pela UFAL - SIMWEAR, além  emprego de perfilagens na construção do modelo em elementos finitos e caracterização estatística do desgaste.

O sistema para geração da chamada folha executiva permitirá especificar a demanda de compra/reserva de tubos e equipamentos de forma integrada aos sistemas de projeto de início de poço (SOLOS e SIMCON) e comunicando-se com outras ferramentas de well supply. Esta integração com outros sistemas de projetos de poços trará conformidade e robustez na troca de informações entre diferentes ferramentas hospedadas no ambiente POÇO WEB, contribuindo com o processo colaborativo que se estende desde os primeiros insumos até a emissão final do projeto.

O processo de instalação do revestimento condutor não é padronizado, seguindo na maioria das vezes a experiência de cada operador.  Dessa forma, o desenvolvimento de estudos experimentais e numéricos, a exemplo dos trabalhos desenvolvidos por P. JeanJean (2009), tornam-se elementos fundamentais para o processo de tomada de decisão. Nesse contexto, no presente projeto propõe-se o desenvolvimento de estudos numéricos ligados as operações de jateamento do revestimento condutor, bem como modelagem de cravação de Base Torpedo, especialmente realizados a partir de métodos meshless como, por exemplo, o Material Point Method.

O simulador SEST SAL permite a previsão do fechamento de poços verticais em função do efeito de fluência da rocha salina na fase de perfuração. Essa previsão auxilia a escolha do peso do fluido de perfuração na fase salina e a decisão acerca de repasses que podem ser necessários. Este simulador está integrado também à ferramenta para acompanhamento da perfuração em tempo real (PWDA). Atualmente o simulador SEST SAL implementa apenas o estágio de fluência secundária, no entanto, diferentes trabalhos são encontrados na literatura confirmando que a maior parte das deformações viscosas das rochas salinas durante a perfuração de poços ocorre nesse estágio. Porém, alguns casos de aprisionamento de broca são reportados instantes após a perfuração de algumas camadas salinas, e a modelagem considerando apenas o estágio secundário de fluência não consegue mapear esses aprisionamentos. Dessa forma, a incorporação de uma lei de fluência primária poderá aumentar a precisão da previsão do fechamento desses poços, identificando esses casos em que a rocha fecha rapidamente nos primeiros momentos após ser perfurada. Propõe-se então a realização de um estudo para identificar uma lei de fluência primária para rochas salinas, e então calibrar seus coeficientes para rochas salinas brasileiras.

Em relação ao comportamento de rochas salinas, propõe-se também o estudo e a incorporação de uma lei de fluência terciária ao simulador SEST SAL, a qual   deve representar o aumento na taxa de deformação e a ocorrência de dano na resposta a longo termo do material.  Além do apoio ao uso do simulador SEST SAL pela equipe do PWDA, um recurso importante é a consideração da lei de fluência primária e fluência terciária discutida. Mas, por atuar no momento da perfuração a lei de fluência primária deve ter maior importância para o PWDA por ocorrer nos primeiros instantes de exposição da rocha. Outras melhorias ou necessidades de ajustes podem ser identificadas ao longo do desenvolvimento do projeto.

Também está previsto a incorporação de um módulo computacional para recomendação de repasses e do peso do fluido de perfuração em rochas salinas. É proposto neste item uma técnica computacional onde automaticamente se experimente diferentes valores de peso de fluido, a partir de um intervalo de valores por exemplo, fornecendo ao usuário informações acerca do número de repasses necessários para os diferentes pesos de fluido considerados. Essa ferramenta deve servir de apoio a decisão sobre a estratégia a ser adotada na perfuração, acelerando o processo de experimentação. Também se propõe no presente projeto a calibração de modelo de fluência salina com retroanálise de prisão de colunas de perfuração na fase salina a partir de dados de falhas operacionais.

Durante a realização de Leak Off Tests (LOT) e após sua finalização (LOT), pode ser possível a ocorrência de crescimento de pressão no anular (Annular Pressure Buildup - APB) devido à elevação de temperatura e à fluência das rochas salinas. A partir de dados de campo, os fenômenos envolvidos devem ser estudados e equações constitutivas devem ser experimentadas. Anomalias foram observadas no APB em regiões salinas onde foram realizados LOT. Nessa etapa se propõe a estudar, desenvolver, e calibrar a partir de dados de campo, uma estratégia para modelagem e previsão desse comportamento. Essa modelagem será incorporada ao simulador computacional de APB já em desenvolvimento no LCCV. Ao final do projeto, a ferramenta computacional desenvolvida a partir desses estudos será incorporado ao POÇO WEB.

O uso frequente dos sistemas SEST SAL, SOLOS E SIMCON nos últimos anos, aliado à perspectiva de aumento constante nessa demanda, bem como a própria evolução do ambiente POÇO WEB, apontam para a necessidade de atualização de alguns recursos e tecnologias computacionais utilizados atualmente. Como exemplo, serão feitos a migração da versão da linguagem de programação Python empregada, a implementação da arquitetura baseada em contêineres (kubernetes, por exemplo) e a mudança no framework para desenvolvimento web.

Por fim, ressalta-se que este projeto visa dar continuidade a estudos e desenvolvimentos iniciados em projetos de P,D\&I anteriores, reafirmando a parceria entre o CENPES/PETROBRAS e o LCCV/UFAL. Este projeto contribui de forma significativa no desenvolvimento de operações de início de poço, perfuração em rochas salinas e na gestão de integridade ao longo da vida operacional, visando atendimento de normativos de segurança operacional, em especial a Resolução ANP n° 46/2016, que institui o Regime de Segurança Operacional para Integridade de Poços de Petróleo e Gás Natural e aprova o Regulamento Técnico do Sistema de Gerenciamento da Integridade de Poços (SGIP).

\section{Objetivo Geral}

O objetivo deste projeto é o desenvolvimento de métodos e ferramentas para avaliação de integridade na fase de perfuração de zonas salinas e para gestão de sistemas de início de poço (solo-revestimento), permitindo a elaboração de projetos de poços de petróleo seguros e confiáveis.

\subsection{Objetivos Específicos}

\begin{itemize}
\item Estudo e desenvolvimento de metodologias para cálculo de capacidade de carga em solos com foco em projeto de início de poço
considerando efeitos de adensamento térmico;

\item  Incorporação da influência na capacidade de carga do solo de efeitos de vibração do motor no condutor jateado;

\item  Atualizações na modelagem de solos considerando incertezas nos dados dos ensaios;

\item Estudos e desenvolvimentos para aplicação de técnicas de aprendizado de máquina na modelagem e caracterização de solos;

\item Análise estrutural e de estabilidade geotécnica de condutores cravados;

\item Otimização de instalação de condutor cravado;

\item Aprimoramento de modelo em elementos finitos para interação de sistema solo-condutor;

\item Aprimoramento de avaliação de desgaste em condutor;

\item Aprimoramento de modelo mecano-fiabilístico no sistema condutor/superfície com base em dados de solos e de fabricação de revestimentos;

\item  Geração automática de folha executiva de início de poço e quantificação de demanda de compra/reserva de tubos e equipamentos da fase de início de poço;

\item Estudos numéricos para otimização de instalação de condutor por jateamento e cravação;

\item Aprimoramento de simulador de fluência em rocha salina no sistema SEST SAL;

\item Elaboração de módulo computacional para recomendação de repasses e do peso do fluido de perfuração em rochas salinas;

\item  Manutenção e melhorias no módulo SEST SAL integrado ao PWDA;

\item Modelagem do Leak off Test (LOT) em regiões com rochas salinas;

\item Calibração de modelo de fluência salina com retroanálise de dados de prisão de coluna de perfuração;

\item  Atualização dos recursos e tecnologias computacionais utilizados no sistema SEST SAL, SIMCON e SOLOS, incluindo atualização da versão do Python, aprimoramento da coordenação da arquitetura baseada em contêineres e mudanças no framework para desenvolvimento web;
\end{itemize}

\section{Resultados Esperados}

\begin{table}[h]
	\begin{tabularx}{\textwidth}{|X|c|}
		\hline
		\rowcolor[HTML]{C0C0C0}
		{\color[HTML]{000000} \textbf{Descrição do Resultado}}  & {\color[HTML]{000000} \textbf{Tipo de Resultado}} \\ \hline
		Desenvolvimento de estudos numéricos ligados a operação instalação de
		revestimento condutor & Conhecimento Produzido \\ \hline
		Desenvolvimento de técnicas para integração entre ferramentas de projetos de poços de petróleo & Método \\ \hline
		Desenvolvimento de sistema de monitoramento integridade de poços na
		perfuração de rochas salinas & Produto \\ \hline
		Disponibilização de módulos computacionais para avaliação de parâmetros do solo
		com foco em projeto de início de poço & Produto \\ \hline
		Disponibilização de módulos computacionais para avaliação do sistema de
		revestimento condutor com base em modelo confiabilístico & Produto  \\ \hline
	\end{tabularx}
\end{table}

%\section{Benefícios para a Indústria}

%Aqui colocar o conteúdo de Benefícios para a Indústria

\section{Mecanismo de Acompanhamento da Execução}

O desenvolvimento do projeto é acompanhado por meio de relatórios técnicos,conforme cronograma definido, descrevendo as atividades realizadas e os resultados obtidos. Reuniões entre a equipe executora, a coordenação do projeto no CENPES/PETROBRAS e clientes finais da PROJ-PERF ocorrem com periodicidade quinzenal, virtualmente, permitindo a avaliação do trabalho realizado. Novas versões das ferramentas computacionais previstas serão disponibilizadas ao longo do projeto, incluindo a incorporação de ajustes pautados pelas demandas e sugestões do corpo técnico e científico da PETROBRAS.

% Metodologia
%\chapter{Metodologia Utilizada}

\setlength{\headheight}{50pt}

Para o desenvolvimento deste projeto, os estudos e desenvolvimentos previstos serão divididos em quatro macroetapas:

\begin{enumerate}

	\item[\textbf{S1}] Estudos e desenvolvimentos relacionados à avaliação de parâmetros de capacidade de carga em solos para projeto de início de poço;Planejamento das atividades e mobilização da equipe

	\item[\textbf{S2}] Estudos e desenvolvimentos relacionados ao projeto de sistema de revestimento condutor e superfície;

	\item[\textbf{S3}] Estudos numéricos para avaliação integrada
	de sistemas de início de poço

	\item[\textbf{S4}] Estudos e desenvolvimentos relacionados a geomecânica de poços em rochas salinas
\end{enumerate}

Essas etapas são indicadas nas atividades a serem desenvolvidas por meio das siglas S1, S2, S3 e S4, respectivamente. Cada etapa é composta por microetapas explicitadas a seguir:

\begin{enumerate}
\item[\textbf{S1.I}] Geração automática de folha executiva de início de poço.

A presente etapa visa o desenvolvimento de metodologia para construção de
um sistema para registro das etapas de início de poço visando a quantificação de demanda de compra/reserva de tubos e equipamentos da
fase de início de poço e consequente integração ao sistema de gerenciamento Cronoweb Materiais;

\item[\textbf{S1.II}] Estudo e desenvolvimento de metodologias para cálculo de capacidade de carga em solos considerando efeitos de adensamento
térmico (poços produtores).

O objetivo dessa etapa é o desenvolvimento e implementação de modelos para avaliar a variação da capacidade de carga em solos usados para assentamento de estruturas de início de poços e sujeitos ao efeito de adensamento térmico, especialmente em poços produtores.

\item[\textbf{S1.III}] Incorporação da influência na capacidade de carga do solo de efeitos de vibração do motor no condutor jateado.

A presente etapa consiste no desenvolvimento de metodologias para incorporação dos efeitos de vibração do motor de fundo empregado em operações de jateamento na capacidade de carga do solo para assentamento do sistema de revestimento condutor;

\item[\textbf{S1.IV}] Cálculo incremental da capacidade de carga do solo ao longo da profundidade com presença de zonas de intercalação de argila/areia.

Nesta etapa, visa-se o desenvolvimento de metodologia para levar em consideração diferentes camadas de argila e areia para avaliação de integridade de sistema de início de poço;

\item[\textbf{S1.V}] Estudos para desenvolvimento e atualização do modelo de cravabilidade em Base Torpedo.

A presenta etapa visa o aprimoramento de metodologias para avaliação de profundidade de cravação em solos cujo início de poço seja realizado por meio de lançamento de base torpedo.

\item[\textbf{S1.VI}] Atualizações na modelagem de solos considerando incertezas dos ensaios.

A presente etapa visa a incorporação de abordagens estatísticas na modelagem de solos, com posterior avaliação da capacidade de carga em solos considerando incertezas calculadas/mensuradas durante a realização de ensaios geotécnicos (como CPTu, por exemplo);

\item[\textbf{S1.VII}] Estudos para aplicação de técnicas de aprendizado de máquina na modelagem de solos.

A presente etapa objetiva a aplicação de técnicas de inteligência artificial para classificação e estimativa da capacidade de carga do solo;

\item[\textbf{S1.VIII}] Atualização dos recursos e tecnologias computacionais utilizados no sistema SEST SOLOS, incluindo atualização da versão da linguagem de programação Python, com implementação da arquitetura baseada em serviços e contêineres, além de uma mudança no framework para desenvolvimento web;

\item[\textbf{S2.I}] Análise estrutural e de estabilidade geotécnica de condutores cravados.

Nessa etapa, serão realizados desenvolvimentos ligados à avaliação de integridade estrutural conjunta solo-revestimento de sistemas condutores instalados por meio de cravação. Também serão considerados: a) Desenvolvimento de estudos e códigos computacionais para modelar o processo de cravação de condutores; b)
Otimização de instalação de condutor cravado, com desenvolvimento de metodologias para otimização do processo de assentamento de revestimento condutor cravado e consequente minimização de tempo de operação de cravação e maximização de ganho de resistência mecânica;

\item[\textbf{S2.II}] Aprimoramento de modelo em elementos finitos para interação sistema solo-condutor

As seguintes atividades são esperadas: i) alteração do elemento finito usado no SIMCON para avaliar rotação na cabeça de poço e integração de carregamentos laterais no projeto; ii) Elemento customizável preenchido pela interface web para inserção de novas curvas P-x e T-z decorrentes das medições;

\item[\textbf{S2.III}] Aprimoramento de desgaste em condutor.

Na presente fase, objetiva-se a integração completa com o sistema SIMWEAR, incluindo a avaliação de perfilagens e caracterização estatística do desgaste, para simular a influência do desgaste na integridade estrutural do conjunto solo-revestimento

\item[\textbf{S2.IV}] Aprimoramento de modelo mecano-fiabilístico com base em dados de solos e cenários de variações na fabricação de revestimentos.

Na corrente fase, está previsto o aprimoramento do modelo de avaliação de risco de falha em sistemas de início de poço considerando as
variabilidades de parâmetros estatísticos de resistência do condutor e parâmetros de resistência de solos;

\item[\textbf{S2.V}]  Atualização e geração automática de relatório de projeto de início de poço.

A presenta etapa visa a atualização de geração de relatório de início de poço com base no método específico empregado para dimensionamento;

\item[\textbf{S2.VI}]  Atualização dos recursos e tecnologias computacionais utilizados no sistema SIMCON, incluindo atualização da versão do Python,
aprimoramento da coordenação da arquitetura baseada em contêineres e mudança no framework para desenvolvimento web

\item[\textbf{S3}]  Estudos numéricos para otimização de instalação de condutor.

Nessa macroatividade, serão realizadas simulações numéricas baseadas
no método dos elementos finitos e em métodos meshless, a exemplo do método dos pontos materiais (MPM) objetivando a otimização de
parâmetros de condutores. Serão considerados os métodos de instalação por cravação e jateamento.

\item[\textbf{S4.I}] Aprimoramento de simulador de fluência salina

Esta etapa se refere à manutenção e incorporação de melhorias do simulador de
fluência salina SEST SAL, especificamente em relação à otimização do código através de duas frentes principais: a) revisão do código em
busca de otimizar seu funcionamento, e b) utilização de modelos axissimétricos híbridos compostos por trechos unidimensionais e
bidimensionais

\item[\textbf{S4.II}] Incorporação de lei de fluência primária e terciária no simulador SEST SAL

Propõe-se nessa etapa a realização de estudos e
desenvolvimentos para identificar uma lei de fluência primária para rochas salinas, e então calibrar seus coeficientes para rochas salinas
brasileiras. Este item prevê também o estudo e a incorporação de uma lei de fluência terciária ao simulador SEST SAL, a qual deve
representar o aumento na taxa de deformação e a ocorrência de dano na resposta a longo termo do material. Assim, esse item é focado no
estudo e experimentação de leis de fluência terciária.

\item[\textbf{S4.III}] Redefinição de elementos devido ao repasse de broca em rocha salina em modelos 2D

Este item envolve o emprego de técnicas de  remalhamento para tornar a operação de repasse mais eficiente numericamente. É proposto elaborar e implementar uma estratégia para remover apenas uma parte do elemento, mantendo seus campos de tensões e deformações inalterados.

\item[\textbf{S4.IV}]  Elaboração de um módulo computacional para recomendação de repasses e do peso do fluido de perfuração em rochas salinas

É proposto neste item inserir uma aba na ferramenta computacional onde automaticamente se experimente diferentes valores de peso de
fluido, a partir de um intervalo de valores por exemplo, fornecendo ao usuário informações acerca do número de repasses necessários para os diferentes pesos de fluido considerados. Essa ferramenta deve servir de apoio a decisão sobre a estratégia a ser adotada na perfuração,  acelerando o processo de experimentação.

\item[\textbf{S4.V}] Elaboração de relatório automático no sistema SEST SAL

Este item trata da elaboração automática de um relatório no formato
definido pela Petrobras com os resultados pertinentes calculados pelo módulo SEST SAL, possibilitando reduzir o trabalho de gerar e
elaborar gráficos e tabelas e organizá-los em um documento. Pretende-se agilizar o processo de emissão do projeto de perfuração,
incrementando, também, iniciativas de auditoria e revisão do mesmo

\item[\textbf{S4.VI}]  Manutenção e melhorias no módulo integrado ao PWDA

Serão realizadas aprimoramentos no simulador SEST SAL com integração ao PWDA, com consideração da lei de fluência primária e terciária definida no presente projeto.

\item[\textbf{S4.VII}] Modelagem do Leak off Test em regiões com rochas salinas

Esse item prevê a modelagem do comportamento de rochas salinas submetidas ao efeito de execução do Leak off Test, incluindo a fase de produção posterior. Haverá Implementação de Leak off Test no simulador de APB. É proposto inserir no simulador de APB já desenvolvido pelo LCCV, a modelagem do LOT. Será desenvolvida uma etapa
anterior a representação do APB na fase de produção, onde haverá a injeção de fluido em um anular específico. Serão implementadas também ferramentas para apresentar os resultados de interesse associados ao LOT. A partir de dados fornecidos pela Petrobras, será feito um estudo comparativo buscando entender o comportamento das formações envolvidas, e levantando possíveis fenômenos (como dano, fratura, porosidade, etc) que expliquem o comportamento observado em campo. A partir do estudo realizado, serão estudadas e
implementadas equações constitutivas adequadas para as formações, de forma a modelar os fenômenos observados.

\item[\textbf{S4.VIII}] Incorporação do simulador computacional LOT ao POÇO WEB

Esse item se refere à incorporação da ferramenta computacional desenvolvida ao POÇO WEB, tornando-a facilmente acessível aos potenciais usuários. Essa incorporação envolve a elaboração de interface para entrada de dados, simulação e apresentação de resultados.

\item[\textbf{S4.IX}]  Calibração de modelo de fluência salina com retroanálise de prisão de coluna de perfuração

Na presente atividade serão desenvolvidas estratégias para calibração de modelos de rochas salinas a partir de dados de falhas operacionais com ocorrência de prisão de coluna de perfuração durante a construção da fase salina.

\item[\textbf{S4.X}]  Atualização dos recursos e tecnologias computacionais utilizados no sistema SEST SAL, incluindo atualização da versão do python, aprimoramento da coordenação da arquitetura baseada em contêineres, e mudança no framework para desenvolvimento web;

\end{enumerate}


% Resultados
%\chapter{Resultados e Discussão}

\setlength{\headheight}{50pt}

Os resultados e discussões das etapas desenvolvidas deste projeto estão apresentadas nos Apêndices de A a L.
%\section{Aferição de Parâmetros do Solo a partir de Modelos Estatísticos}

%Ao se deparar com meios altamente heterogêneos, análises estatísticas e estudos sobre a variabilidade de parâmetros auxiliam na tomada de decisão com confiabilidade. Sendo assim, uma metodologia para estimativa de parâmetros de solo a partir de modelos estatísticos para aplicação em projetos de poços petróleo, focando na fase inicial de execução, é apresentada no apêndice \ref{chap:estatistica}.

%A partir desta metodologia, determinada através dos estudos das normas e da literatura, dados de 16 ensaios CPTu foram analisados. Como resultado, pontua-se que a recomendação da NORSOKG-001 retorna valores característicos mais altos, Lacasse et al. \cite{lacasse2007statistical} valores localizados como um limite inferior da NORSOKG-001 e a DNV-RP-C207 apresenta um desempenho dependente do parâmetro geotécnico analisado.

%Tratando-se de classificação do solo a partir do CPTu, a metodologia de Jefferies \& Davies \cite{jefferies1993use} fornece resultados satisfatórios. Sendo esta a escolha para as futuras etapas do trabalho.

%\section{Extrapolação de Dados de Resistência Não Drenada}

%No apêndice \ref{chap:extrapolacao}, motivado pela necessidade da obtenção dos valores de resistência do solo em profundidades superiores às alcançadas pelo ensaio de piezocone, metodologias de extrapolação da resistência não drenada são selecionadas. Tais estratégias consistem em métodos analíticos e redes neurais, que é um método de inteligência artificial. Para a classificação do solo utilizou-se o trabalho de Jefferies \& Davies \cite{jefferies1993use}, escolha baseada no resultado obtido em estágio anterior desta pesquisa.

%Como resultado, as metodologias analíticas se mostraram mais considerando o custo computacional e o erro das soluções comparadas. Vale ressaltar também que a heterogeneidade e a qualidade dos dados a serem extrapolados tem relação direta com o resultado.

%\section{Metodologia para Estimativa Espacial de Parâmetros do Solo com Base em Geoestatística}

%O estudo de confiabilidade fornece ao tomador de decisões ferramentas para lidar racionalmente com incertezas presentes na execução de projetos. Sendo assim, no apêndice \ref{chap:geoestatistica} é apresentado uma metodologia baseada em geoestatística para a estimativa e análise de parâmetros de solo.

%Após a revisão bibliográfica, onde os principais conceitos e formulações sobre variogramas e krigagem foram apresentados, iniciou-se a fase de implementação dos principais semivariogramas e ajustes com o intuito de entender a influência dos poços vizinhos no poço virtual. Em seguida, os modelos de krigagem foram desenvolvidos e validados. Tais etapas são de extrema importância para a geração dos perfis virtuais capazes de auxiliar corretamente as decisões de projeto.

%\section{Módulos para Avaliação de Parâmetros de Solo para Integração no Sistema PoçoWeb}

%Na presente etapa descrita no anexo \ref{chap:codigo}, o código computacional desenvolvido para a análise e cálculo dos indicadores utilizados no projeto de poço é destrinchado. Além do diagrama UML código, esta seção serve também como uma documentação para consulta, uma vez que explica todas as funções e atributos da classe CPT.

%\section{Modelagem Computacional de Cravação de Revestimento Condutor}

%A cravação consiste em um estágio inicial para a instalação do revestimento condutor. O apêndice \ref{chap:cravacao} é destinado para a apresentação de uma revisão bibliográfica e uma metodologia para a modelagem do procedimento, bem como a reprodução de modelos presentes na literatura através do software ABAQUS.

%A modelagem da cravação é dividida em duas etapas: cravação por peso próprio e cravação por martelamento. A cravação por peso próprio é sensível a primeira resistência de contato entre o solo e o condutor, já por martelamento, o esforço aplicado no condutor vem através de carregamentos de ondas.

%Como resultado, uma metodologia a ser seguida foi apresentada, da mesma maneira que o modelo constitutivo que representará o solo foi definido: Mohr-Coulomb. Por se tratar de um caso que lida com grandes deformações e grandes deslocamentos, duas abordagens são possíveis de serem adotadas (CEL ou ALE) para a simulação computacional, tal decisão dependerá de qual a magnitude do caso a ser simulado. Sendo assim, a próxima demanda será o desenvolvimento da modelagem da operação de Papa-Terra.

%\section{Modelagem Numérica de Jateamento}

%Além da instalação do revestimento condutor por cravação, outra técnica amplamente utilizada é o jateamento. A instalação por jateamento é geralmente realizada em solos com camadas de sedimentos não consolidados. Com o impacto do jato estes sedimentos são carreados, abrindo espaço para a passagem do condutor.

%No apêndice \ref{chap:jateamento} uma revisão bibliográfica com foco na modelagem do problema é apresentada, seguida de uma metodologia e da modelagem de estudos de casos baseados na literatura. Tais modelagens foram executadas no software XFLOW, que trata a fase fluida através do Método de Lattice Boltzman e no ABAQUS.

%Duas estratégias diferentes foram utilizadas para a representação do solo, a primeira tratando o como um fluido altamente viscoso e a segunda através de uma abordagem sólida. Como resultado chegou-se a conclusão de que apesar da abordagem fluida apresentar resultados promissores, a validação do modelo não é possível com os dados que a equipe executora possui. Sendo assim, neste trabalho o solo será modelado como sólido.

%\section{Modelagem Mecano-fiabilística aplicada à Projeto de Poço}

%O revestimento condutor exerce papel fundamental na transmissão de esforços da cabeça de poço para o solo. Sendo assim, é de extrema valia avaliar integridade do sistema solo-revestimento para a elaboração de poços seguros.

%O anexo \ref{chap:confiabilidade} apresenta uma revisão bibliográfica que embasa o desenvolvimento de um modelo mecano-fiabilístico. Tal etapa tem o intuito de fundamentar conceitos básicos a fim de facilitar o avanço na integração do SIMCON com as ferramentas do Poço Web. As demais atividades após a conclusão da revisão estão descritas com mais detalhes no anexo.

%\section{Análises da interação Solo-Revestimento em Método dos Elementos Finitos}

%O apêndice \ref{chap:setup} é direcionado para o estudo da interação direta entre solo e condutor. Tal etapa é de suma importância devido a natureza do procedimento. Após a cravação ou jateamento o solo sofre significativa deformação, alterando sua condição inicial e consequentemente causando um rearranjo na distribuição do poro pressão, capaz de influenciar na capacidade de carga com o tempo. Este efeito denomina-se Setup.

%Para que seja possível a simulação computacional do problema, uma revisão bibliográfica detalhada e uma metodologia para a execução do modelo é apresentada. As simulações, que estão em andamento e serão documentadas nos próximos relatórios, serão realizadas tanto para a cravação quanto para o jateamento.

%\section{Atividades Complementares}

%Para suporte da compreensão dos fenômenos físico-químicos de formação do solo na região de interesse, foi gerado o apêndice \ref{chap:bacias}.

\section{Trabalhos e Publicações}

Dentro das necessidades acadêmicas do grupo, bem como parte da busca por um melhor resultado na entrega dos produtos, publicações dos trabalhos executados servem como formas de contribuição dentro da comunidade acadêmica, além da apresentação de habilidades e \textit{expertise} dos centros de pesquisa envolvidos. Nesse sentido, a Tabela \ref{tab::published_work} traz a lista dos trabalhos aprovados e aguardando aceite até Fevereiro de 2024.

\begin{center}
    \begin{xltabular}{\textwidth}{|X|X|X|}
        \hline
        \textbf{Trabalho}  & \textbf{Congresso} & \textbf{Data Apresentação}  \\ \hline
        \endfirsthead

        \hline
        \textbf{Trabalho}  & \textbf{Congresso} & \textbf{Data Apresentação}  \\ \hline
        \endhead

        Análise da Integridade do Revestimento e Sistema de Cabeça de Poço em Cenário de Worst Case Discharge.  & ENAHPE 2023 & Agosto/2023 \\ \hline
        Abordagens de Otimização para Início de Poço: Um estudo de caso em Bacia da Costa Leste Brasileira.  & ENAHPE 2023 & Agosto/2023 \\ \hline
        Análise Confiabilística da Influência da Resistência na Avaliação de Critérios de Projeto Estrutural de Poços de Petróleo. & ENAHPE 2023 & Agosto/2023 \\ \hline
        Modelagem Numérica e Computacional do Jateamento de Revestimento Condutores: o estado da arte & ENAHPE 2023 & Agosto/2023 \\ \hline
        Modelagem da cravação do revestimento condutor com o MPM & RIO OIL \& GAS 2024 & Setembro/2024 \\ \hline
        Thermomechanical modeling of the leak off test (LOT) in oil wells in the presence of evaporites & RIO OIL \& GAS 2024 & Setembro/2024 \\ \hline
        On The Probabilistic Assessment Of Top-hole Casing Design & Offshore Technology Conference 2024 & Maio/2024 \\ \hline
        Bayesian-Based Approach in Soil Characterization for Top-hole Design & Offshore Technology Conference 2024 & Maio/2024 \\ \hline
        Parametric Study of Conductor Casing Installation Using Hydraulic Hammering Method: A Numerical Approach with Material Point Method & XLV CILAMCE & Novembro/2024 \\ \hline
        Numerical Modelling of Conductor Casing Settlement Using a Two-phase Model & XLV CILAMCE & Novembro/2024 \\ \hline
        A Numerical Investigation of Conductor Casing Installation Using Material Point Method & XLV CILAMCE & Novembro/2024 \\ \hline
        Influence of velocity on conductor casing driving via Material Point Method & XLV CILAMCE & Novembro/2024 \\ \hline
        On the probabilistic assessment of casing applied to top hole design by FORM & XLV CILAMCE & Novembro/2024 \\ \hline
        Computational Modeling of Torpedo Anchor Penetration at Seabed Using Piezocone Tests & XLV CILAMCE & Novembro/2024 \\ \hline
        Implementation of a design methodology for well foundation in case of conductor with insufficient axial resistance & XLV CILAMCE & Novembro/2024 \\ \hline
        Local remesh procedure to model reaming in vertical oil wells drilled through salt rocks & XLV CILAMCE & Novembro/2024 \\ \hline
        Predictive Characterization of Fracture and Absorption Tests in Halite Formations: an integrated approach of numerical modeling and field data & XLV CILAMCE & Novembro/2024 \\ \hline
        Modeling of Creep Closure of Salt Rocks Drilled by Directional Wells & XLV CILAMCE & Novembro/2024 \\ \hline
        Comparative analysis of wellbore closure in salt rock formations considering primary creep & XLV CILAMCE & Novembro/2024 \\ \hline
        Assessing the Impact of Setup Effect on Top-Hole Design: A Probabilistic Approach (Em processo de submissão) & Offshore Technology Conference 2025 & Maio/2025 \\ \hline
        Application of Data-Driven Techniques in 3D Soil Characterization for Top-Hole Design (Em processo de submissão) & Offshore Technology Conference 2025 & Maio/2025 \\ \hline
        Numerical Modeling of Conductor Casing Installation: Leveraging the Material Point Method (Em processo de submissão) & Offshore Technology Conference 2025 & Maio/2025 \\ \hline
    \end{xltabular}
    \captionof{table}{Lista de trabalhos aprovados, submetidos e aguardando avaliação e suas respectivas datas de apresentação.}
    \label{tab::published_work}
\end{center}

Além das publicações feitas nos mais diversos congressos, também foram realizadas publicações em algumas revistas, as quais podem ser encontradas abaixo:

\begin{itemize}
	\item VÁRADY FILHO, C. A. F. et al. On the Probabilistic Assessment of Tophole Casing Design. SPE Journal, v. 29, n. 09, p. 4764-4770, 2024.
    \item VÁRADY, C. et al. Bayesian-Based Approach in Soil Characterization for Tophole Design. SPE Journal, v. 29, n. 11, p. 5792-5803, 2024.
\end{itemize}

% Considerações
%% \begin{comment}%\chapter{Considerações Parciais/Finais e Recomendações}

%Conforme descrito, a execução do projeto foi subdividida em etapas associadas a um cronograma de execução. Dessa forma, tem-se a cada relatório um acompanhamento da evolução de desenvolvimento por meio de um percentual de execução. De forma resumida, serão descritas nesse capítulo as considerações relativas a cada etapa de desenvolvimento executada dentro do período definido para esse relatório. Ressalta-se que, conforme apresentado no Capítulo anterior, os resultados e discussões para as atividades realizadas estão no Apêndices seguintes.

%Para o presente relatório, conforme indicado no cronograma, ficou determinada a execução das atividades A0.1, A0.2, A0.3, A1.1, A1.2, A1.3, A1.4, A1.5, A3.1, A3.2, A4.1, A4.2 e A6.

%\section{Mobilização da equipe, planejamento das atividades e reunião de abertura}

%A equipe contratada é apresentada na Tabela \ref{tab::equipe} e é composta por um coordenador, 16 pesquisadores e 4 bolsistas de graduação. A equipe foi subdividida de acordo com cada etapa do projeto.

%\begin{table}[H]
%	\centering
%	\begin{tabularx}{\hsize}{|l|l|l|l|l|} 
%		\hline
%		\multicolumn{1}{|c|}{\textbf{Nome}} & \multicolumn{1}{X|}{\textbf{Período (em meses)}} & \multicolumn{1}{X|}{\textbf{Carga Horária semanal}} & \multicolumn{1}{X|}{\textbf{Função}} & \multicolumn{1}{X|}{\textbf{Vínculo principal}} \\ \hline
%		João Paulo Lima Santos & 36 & 2 & Coordenador & UFAL \\ \hline
%		Aline da Silva Ramos Barboza & 36 & 2 & Pesquisador & UFAL \\ \hline
%		Eduardo Toledo de Lima Junior & 36 & 2 & Pesquisador & UFAL \\ \hline
%		Eduardo Nobre Lages & 36 & 2 & Pesquisador & UFAL \\ \hline
%		Jose Luis Gomes Marinho & 36 & 2 & Pesquisador & UFAL \\ \hline
%		Luciana Correia Laurindo Martins Vieira & 36 & 2 & Pesquisador & UFAL \\ \hline
%		Lucas Pereira de Gouveia & 36 & 2 & Pesquisador & UFAL \\ \hline
%		Beatriz Ramos Barboza & 18 & 40 & Pesquisador &  LCCV\\ \hline
%		Christiano Augusto Ferrario Várady Filho & 36 & 40 & Pesquisador & LCCV \\ \hline
%		Jennifer Mikaella Ferreira Melo & 18 & 20 & Pesquisador & LCCV \\ \hline
%%		Joyce Kelly França Tenório & 18 & 20 & Pesquisador & LCCV \\ \hline
%		Raniel Deivisson de Alcantara Albuquerque & 18 & 20 & Pesquisador & LCCV \\ \hline
%		Natália de Carvalho Souza dos Santos & 18 & 20 & Bolsista-Graduação & LCCV \\ \hline
%		Júlia Beatriz Ferreira Souza & 18 & 20 & Bolsista-Graduação & LCCV \\ \hline
%	\end{tabularx}
%\caption{Equipe do projeto.}
%\label{tab::equipe}
%\end{table}

%O planejamento das atividades refletiu-se na execução do cronograma e a reunião de abertura foi realizada em 01 de outubro de 2018.

%\section{Aferição de parâmetros do solo a partir de modelos estatísticos}

%O processo de aferição objetiva a definição dos valores que caracterizam de maneira precisa o comportamento mecânico do solo. Nesse aspecto, a metodologia de aferição seleciona os dados mais interessantes e retira fontes de ruídos (também denominados \textit{outliers} por se localizarem fora da curva modelada do comportamento). 

%Todo o processo de revisão bibliográfica, aplicação de metodologia para caracterização estatística de parâmetros do solo ao longo da profundidade, bem como a realização de ajustes com eliminação de \textit{outliers} seguindo a metodologia indicada pela DNV-RP-C207 \cite{dnvrpc207} e baseando-se em intervalo de confiança foi executada. Ressalta-se que também foi inserido um método numérico para classificação do solo que apresentou resultados muito próximos dos apresentados fornecidos pela empresa (RAGIP). Ressalta-se que as formulações foram implementadas em algoritmo computacional e resultados referentes aos campos de Búzios e Tartargura Mestiça foram avaliados. Maiores informações sobre a teoria e os resultados podem ser conferidas no Apêndice \ref{chap:estatistica} e sobre o código no Apêndice \ref{chap:codigo}.

%\section{Metodologia para estimativa espacial de parâmetros de solo com base em geoestatística}

%O emprego da geoestatística visa a estimativa de parâmetros geotécnicos para determinado local baseando-se nas informações de outros furos ponderadas a partir da sua localização geográfica. A revisão de literatura, com aplicação da formulação clássica dos modelos de correlação, montagem de semivariogramas experimentais (conforme \citet{matheron1963}) e robustos (conforme \citet{cressie} e \citet{dowd}), além da formulação de krigagem ordinária e universal já estão delimitadas e funcionais. Em outro aspecto, a geração do perfil geotécnico de um poço virtual está desenvolvida e os primeiros resultados estão incorporados ao Apêndice \ref{chap:geoestatistica}. Os procedimentos de correlação, geração do semivargiograma e ajustes esférico, exponencial e potencial foram implementados. Toda formulação foi implementada em ambiente computacional desenvolvido em linguagem Python, cuja documentação está inserida no Anexo \ref{chap:codigo}.

%\section{Módulos para avaliação de parâmetros do solo para integração do sistema PoçoWeb}

%As metodologias de aferição e estimativas são implementadas em um módulo computacional para análise e cálculo de parâmetros para quaisquer ensaios CPTs de furos inseridos no sistema. Basicamente, é o produto a ser utilzado. Um algoritmo computacional para aferição de parâmetros de solo usando técnicas estatísticas e geoestatísticas está em fase final de testes com todos os dados geotécnicos enviados. Até o presente momento, o código apresenta a estrutura modular de uma biblioteca em Python, já executando os procedimentos estatísticos e geoestatísticos propostos para o presente projeto. Nesse aspecto, o leitor é direcionado para os Apêndices \ref{chap:estatistica}, \ref{chap:geoestatistica} e \ref{chap:codigo} para maiores informações. 

%\section{Revisão Bibliográfica sobre jateamento e cravação de sistemas de revestimento condutor}

%O estudo de técnicas de modelagem computacional de processo de perfuração prenunciam a própria modelagem do fenômeno estudado. A revisão de literatura sobre jateamento encontra-se finalizada, com separação de modelagens de interesse ao processo de análise fluidodinâmica, a exemplo do modelo desenvolvido por \citet{wang2014numerical}. A revisão de literatura para modelos de cravação também encontra-se finalizada. Informações técnicas detalhadas sobre as revisões podem ser encontradas nos Apêndices \ref{chap:cravacao} e \ref{chap:jateamento}.  

%\section{Modelagem Numérica da Cravação de Revestimento Condutor}

%Atualmente, a modelagem está sendo realizada aplicando a técnica numérica de acoplamento de malhas eulerianas e lagrangeanas (\textit{Coupling Eulerian Lagrangian Technique} - CEL). A integração entre as malhas, com indicativo de proporcionalidade do volume total de solo estudado, permite que a captação de dados da modelagem levando em consideração a parcelas sólidas (particuladas) e líquida (poropressão) do solo como um todo. Nesse sentido, o modelo constitutivo adotado para a representação do solo é Mohr-Coulomb. As modelagens estão em fase de validação e aplicação no modelo de cravação. Maiores informações sobre os resultados e sua discussão estão inseridos no Apêndice \ref{chap:cravacao}.

%\section{Modelagem Numérica de Operações de Jateamento}

%A partir da revisão de literatura sobre cravação de revestimento condutor por jateamento, algumas análises iniciais foram realizadas no Abaqus com o principal objetivo de compreensão do processo de modelagem. Um modelo de acoplamento sólido-fluido foi montado tendo como dados de entrada campos de velocidade e pressão para a fase fluida, e dados de tensão para a região sólida. Adicionalmente, foi fornecido pela Petrobras um modelo do domínio fluido de uma broca tricônica, que serviu para a geração de fluxos em regimes linear e turbulento empregando a modelagem $k-\epsilon$ e Spallart-Allmaras para regimes turbulentos. 


%Na literatura é possível encontrar duas modelagens distintas para a dinâmica de solos: utilizando equacionamento proveniente da dinâmica dos fluidos ou atráves da mecânica dos sólidos tradicional. Após estudos e experimentações sobre qual a abordagem melhor encaixaria aos objetivos do projeto, optou-se por tratar a problematica da cravação por jateamento com o solo sendo modelado pela mecânica do sólidos tradicional. O jato, por sua vez, seguiu sendo simulado atráves da equação de Boltzmann. Maiores informações sobre os resultados e discussões obtidos estão inseridos no Apêndice \ref{chap:jateamento}.
% \end{comment}

% Introduzir aqui os capítulos adicionais do documento
% \input{texto/nomedocapitulo} %criar o arquivo nomedocapitulo.tex na pasta includes

% Apêndices, NÃO MODIFICAR A LINHA ABAIXO
\appendix

%\chapter{SEST Solos: Alterações e Implementação de novas funcionalidades}
%\label{chap:SestSolos}
%\input{texto/chapter_solos}

%\chapter{SIMCON: Alterações e Implementação de novas funcionalidades}
%\label{chap:Simcon}
%\input{texto/chapter_simcon}

%\chapter{SEST SAL: Alterações e Implementação de novas funcionalidades}
%\label{chap:SestSal}
%\input{texto/chapter_sestsal}

\chapter{Modelagem Computacional de Cravação de Revestimento Condutor}
\label{chap:cravacao}
\section{Considerações Iniciais }

A etapa inicial da perfuração de um poço de petróleo ocorre com a instalação do revestimento condutor, ou estrutural, no solo. Conforme sugere, a função deste tubo é estabilizar as paredes do poço e fornecer suporte aos equipamentos de cabeça e aos revestimentos das fases subsequentes. A sua instalação o varia com a localização do poço (\textit{onshore} ou \textit{offshore}) e as características geotécnicas do solo.  

No cenário \textit{offshore}, a cravação é uma alternativa recorrente para o início de poço. O processo que levará à execução dessa atividade consiste de certas etapas, dentre elas, o ajuste de parâmetros de projeto, muitas vezes, realizado com base em ensaios experimentais. No entanto, em lâminas d’água elevadas, a execução desses ensaios tende a ser arriscada e dispendiosa, estendendo cronogramas. Para mitigar custos e riscos, estudos numéricos apoiados em modelos validados e calibrados oferecem uma representação confiável da instalação do condutor no solo.

O presente apêndice reúne os resultados da modelagem numérica da cravação de um revestimento condutor, considerando um domínio em dimensões e dados reais. O desenvolvimento foi suportado por dados de solo do campo Papa Terra, abrangendo relatórios de  instalação por martelamento e resultados de sondagens CPTu realizados nesse campo.

Como é sabido, problemas de impacto deste tipo envolvem grandes deformações e, ao serem modelados numericamente, resultam em grandes distorções de malha computacional se solucionados por métodos numéricos baseados na mecânica do contínuo.  Como alternativa, vêm ganhando destaque na comunidade geotécnica os métodos numéricos baseados em partículas, com destaque para o Método dos Pontos Materiais (MPM), particularmente adequado para problemas com grandes deformações. Assim, o MPM se apresenta como uma abordagem promissora para a modelagem desta operação.


\section{Modelagem numérica}

Para atender ao desafio proposto, a análise numérica foi realizada no \textit{software} \textit{open source} ANURA3D, que utiliza o Método dos Pontos Materiais para processar os cálculos. Na sequência, apresentam-se, de forma sucinta, as formulações e os algoritmos adotados, bem como a descrição da geometria, das condições de contorno e das propriedades dos materiais.

\subsection{Método dos Pontos Materiais}

O Método dos Pontos Materiais (MPM) é uma abordagem híbrida Euleriana-Lagrangiana que combina as vantagens de métodos numéricos baseados em partículas e baseados em malha. Nesse arcabouço, o domínio computacional é discretizado por uma malha Euleriana de fundo fixa. Ao mesmo tempo, os corpos materiais são representados por partículas Lagrangianas (pontos materiais) que transportam todas as propriedades físicas e constitutivas (massa, densidade, tensões, deformações, etc.) e se deslocam através da malha. Essa dupla caracterização permite ao MPM lidar naturalmente com grandes deformações, evitando os problemas de distorção de malha inerentes aos métodos puramente Eulerianos (\citet{yerro2022modelling}; \citet{nguyen2023material}). A Figura \ref{fig:MPM} mostra um breve esquema do ciclo de simulação numérica utilizando o MPM.

\begin{figure}[H]
	\centering
  	\includegraphics[width=1\linewidth]{cravacao/MPM2.pdf}
	\caption{Algoritmo de resolução do MPM. Fonte: Autores (2026)}
	\label{fig:MPM}
\end{figure}

O método impõe a conservação da quantidade de movimento, conforme expresso na Eq.\eqref{eq:quantidade_movimento}. A integração numérica prossegue pelas etapas: cálculo das funções de forma para todos os pontos materiais ao longo da malha de fundo e transferência das propriedades de massa e quantidade de movimento das partículas para os nós (\ref{fig:MPM}a); solução das equações de quantidade de movimento na malha (Eqs.\eqref{eq:forca_interna} e \eqref{eq:forca_externa}) (\ref{fig:MPM}b); e projeção das velocidades e posições nodais atualizadas de volta aos pontos materiais (\ref{fig:MPM}c). Onde $\sigma_p$ é a tensão de Cauchy; $\nabla N_I(\mathbf{x}_p)$ é o gradiente da função de forma; $V_p$ e $\rho_p$ são, respectivamente, o volume e a densidade do ponto material; $\mathbf{b}$ é a força de corpo; e $\mathbf{f}_I^{int}$, $\mathbf{f}_I^{ext}$ são as forças nodais internas/externas. Ao final de cada passo de tempo, a malha é reestruturada para a posição inicial e as partículas permanecem na posição deslocada, prevenindo distorções excessivas nos elementos (\citet{martinelli2020numerical}, \citet{yost2022}, \citet{fu2024material}).

\begin{equation}
	\rho \frac{dv}{dt} = \nabla \cdot \sigma + \rho \mathbf{b}
	\label{eq:quantidade_movimento}
\end{equation}

\begin{equation}
	f_I^{\text{int}} = - \sum_p \sigma_p \,\nabla N_I(\mathbf{x}_p)\, V_p
	\label{eq:forca_interna}
\end{equation}

\begin{equation}
	f_I^{\text{ext}} = \sum_p N_I(\mathbf{x}_p)\,\rho_p\,\mathbf{b}\, V_p
	\label{eq:forca_externa}
\end{equation}

\subsection{\textit{Two-phase single point formulation:} modelagem para problemas de interação solo-água}

A formulação \textit{Two-phase single point}, em modo explícito no MPM, é capaz de descrever como ocorre a interação solo-água, levando-se em consideração a geração e a dissipação de pressão de poros mediante a descida da estrutura no solo, assim como a propagação das ondas dinâmicas em ambas as fases \citet{alkafaji2013}. Nesta formulação todas as fases (por exemplo, esqueleto sólido, água e, eventualmente, ar) são representadas por um conjunto único de pontos materias. Os pontos materiais se movem juntamente com o movimento sólido (descrição lagrangiana), enquanto os fluidos são descritos usando um referencial euleriano. Essa representação dos meios multifásicos é conveniente para muitas aplicações geotécnicas, mas não considera a separação física entre água livre e esqueleto sólido, como é no caso da aplicação no presente modelo \citet{ceccato2024simulating}. As equações de governo são conservação de massa, conservação de momento e um modelo constitutivo capaz de prever a solução do sistema, descritas as equações 1 a 5, conforme descrito por \cite{martinelli2022explicit}:


\begin{equation}
	\rho_L \textbf{a}_L = \nabla \cdot (p_L\textbf{}I)+  \rho_L \textbf{b}-\frac{n\mu_L}{\kappa_L} (\nu_L - \nu_S)
\end{equation}

\begin{equation}
	(1-n)\rho_S\textbf{a} + n\rho_L\textbf{a} = \nabla \cdot (\sigma'+p_L\textbf{I})+\rho_m\textbf{b}
\end{equation}

\begin{equation}
	\frac{dn}{dt}=(1-n)(\nabla \cdot \nu_S)
\end{equation}

\begin{equation}
	\dot{P} = \frac{K_L}{n}[(1-n)\nabla \cdot \nu_S + n\nabla \cdot \nu_L]
\end{equation}

\begin{equation}
	\dot{\sigma} = D\dot{\varepsilon}_S - \Omega\sigma' - \sigma'\Omega - \dot{\varepsilon}_{\nu,S} \sigma'
\end{equation}.

Onde, $n$ é a porosidade; $\rho_S$ e $\rho_L$ são as densidades dos grãos e da fase líquida, respectivamente; $\rho_m = (1-n)\rho_S + n\rho_L$ é a densidade da mistura; \textbf{I} é o tensor identidade; $\nu_S$ e $\rho_L$ são, respectivamente, as velocidades das fases sólidae líquida; $\sigma'$ é o tensor de tensões; $K_L$ é o módulo volumétrico da água pura; $\textbf{b}$ é o vetor de força do corpo; $\textbf{D}$ é a matriz de rigidez; $\cdot{\sigma'}$ e $\dot{\varepsilon}$ são, respectivamente, as taxas de tensão e de deformação da fase sólida; $\Omega$ é o tensor de rotação e $\varepsilon_{\nu,s}$ é o incremento de deformação volumétrica \citep{martinelli2022explicit}. As equações 6 e 7 contêm, respectivamente, a forma discretizada das equações de balanço 1 e 2 para um nó ativo genérico da malha computacional.

\begin{equation}
	\bar{M}_{L,i}=\bar{f}^{ext}_{L,i} - \bar{f}^{int}_{L,i} + \bar{f}^d_i
\end{equation}

\begin{equation}
	\bar{M}_{S,i}a_{S,i} + \bar{M}_La_{L,i} = \bar{f}^{ext}_i - \bar{f}^{int}_i 
\end{equation}


Nas quais, $\bar{M}_{S,i}$, $\bar{M}_{L,i}$, $\bar{f}^{ext}_{L,i}$, $\bar{f}^{int}_{L,i}$, $\bar{f}^d$ são os valores nodais, respectivamente, para: a matriz de massa das fases sólida e líquida; os vetores de força externa, de força interna e de força de arrasto. Estas forças são dependentes do número de elementos ao redor do nó, do número de pontos materiais em cada elemento de malha, da função de forma calculada para a posição de ponto material e da força gravitacional.

\subsection{Algoritmo de contato}
Na Engenharia Geotécnica, é comum lidar com problemas que envolvam interação solo-estrutura. Quando um deslizamento por atrito ocorre na superfície de contato entre estes corpos, necessita-se de um algoritmo específico, que permita o movimento relativo entre eles. O algoritmo de contato utilizado foi orginalmente desenvolvido por \citet{bardenhagen2001improved}, que o formulou para modelar tanto a separação entre os corpos quanto o contato com deslizamento por atrito entre eles. Posteriormente, este algoritmo foi extendido a contatos adesivos por \citet{alkafaji2013} \citep{martinelli2021investigation}, como é o caso dos solos coesivos não drenados. A vantagem desse algoritmo é que ele detecta a superfície de contato automaticamente, ou seja, não é necessário definir nenhum elemento especial na interface entre os corpos \citep{alkafaji2013}.

Inicialmente, o efeito do contato é realizado pela correção da diferença de velocidade entre os corpos: seja um sistema com dois corpos A e B em contato de deslizamento, cujas velocidades prescritas individuais (${\nu}_{k,a}$ e ${\nu}_{k,b}$) e a velocidade combinada do sistema $({\nu}_{k,s})$ são resolvidas através das equações de movimento para um nó de contato $k$. Tomando a velocidade prescrita do corpo A num instante de tempo $t$+$\Delta t$:

\begin{equation}
	{\begin{matrix}
			(\nu_{k,A}^{t+\Delta t}-\nu_{k,S}^{t+\Delta t}) \cdot n_k^t>0 
			\\
			(\nu_{k,A}^{t+\Delta t}-\nu_{k,S}^{t+\Delta t}) \cdot n_k^t<0
		\end{matrix})}
\end{equation}
	
	\subsection{Algoritmo do corpo rígido}
	
	O esquema de integração explícita é condicionalmente estável. O tamanho do passo de tempo para uma solução estável decresce com o aumento da rigidez do material e com a diminuição do tamanho do elemento da malha computacional \citep{martinelli2020numericalm}. A fim de reduzir o custo computacional, o condutor é modelado como um corpo rígido através do algoritmo desenvolvido por \citet{zambrano2020numerical}. Esta aproximação desconsidera a propagação de onda na estrutura e torna-se válida porque a rigidez do aço do condutor é muito maior se comparada à rigidez do solo \citep{galavi2019}.
	
	\subsection{Malha móvel}
	
	Com o intuito de preservar os elementos de malha ao redor da estrutura da deformação imposta, utiliza-se o procedimento de malha móvel. A malha móvel se aproveita do fato da malha computacional não armazenar informações permanentes, já que elas são transmitidas aos pontos materiais Cecatto (2022). Este procedimento consiste basicamente em ajustar a malha adjacente à estrutura ao seu movimento após cada passo de tempo, assegurando que a sua superfície coincida com os elementos de fronteira e, com isso, os elementos manterão a mesma forma durante a simulação \citep{ceccato2017numerical}: a malha móvel será a porção da malha cujos elementos não se deformarão; por outro lado, a porção da malha localizada entre a malha móvel e as restrições do domínio irá se deformar.
	
	\subsection{Histórico de desenvolvimento do modelo numérico atual}
	
	Para que a geometria do sistema seja devidamente representativo, deve-se construir um modelo numérico característico da situação real do problema proposto, considerando-se as variáveis de projeto, as restrições adotadas e o tamanho ideal do domínio, que seja suficiente para evitar efeitos de borda,  mas que tenha o menor custo computacional possível sem comprometer a qualidade dos resultados da simulação. Ao calibrar um modelo numérico, por muitas vezes, deve-se adaptar a geometria e realizar o remalhamento da malha computacional. 
	
	O modelo numérico até então desenvolvido conseguiu cravar até a metade da segunda camada de solo, totalizando cerca de 13,5 m de profundidade a partir da \textit{mudline}. Depois disso, a simulação abortou por problemas que, ainda nos dias de hoje, estão sob análise. Provavelmente, ainda são problemas numéricos relacionados às partículas nas regiões de \textit{shoulder} (o encontro entre a ponta e a base do condutor) e ponta do condutor, onde ocorrem grandes concentrações de tensão. No "\textit{shoulder}" é comum que um fenômeno numérico indesejado, a "interpenetração", ocorra. Nele duas ou mais partículas materiais ocupam o mesmo espaço simultaneamente, ou seja, suas fronteiras se sobrepõem. A interpenetração se dá, por exemplo, devido a uma deformação extrema nos corpos, é o que se observa no problema proposto, dado que há uma estrutura rígida penetrando um corpo altamente deformável. Ou, então, devido a erros numéricos, como a discretização inadequada das equações de movimento. 
	
	No caso em questão, provavelmente ocorreu porque no vértice do \textit{shoulder}, estavam constando dois vetores normais, que "confundiam" o movimento das partículas, resultando na interpenetração, conforme mostra a Figura \ref{fig:mudanca_ponta}. Um dos pontos revisados foi o remalhamento e a alteração no contato solo-estrutura.  Salienta-se que ao, realizar a suavização do \textit{shoulder}, é necessário se atentar ao refinamento da malha nesta região. Um refinamento muito excessivo pode elevar bastante o custo computacional da simulação. A Figura \ref{fig:mudanca_ponta}a ilustra a configuração da ponta deste modelo numérico, um um \textit{shoulder} mais acentuado e a ponta menos refinada. A Figura \ref{fig:mudanca_ponta}b já mostra a nova configuração desta mesma região que, embora apresente uma mudança sutil, melhorou consideravelmente os resultados.
	
	\begin{figure}[H]
		\centering
		\includegraphics[width=0.7\linewidth]{cravacao/mudanca_ponta.pdf}
		\caption{Breve descrição da transição realizada na ponta do condutor. Fonte: Autores (2026)}
		\label{fig:mudanca_ponta}
	\end{figure}

Em relação ao último resultado, o perfil de descida do condutor ao longo do tempo pode ser analisado na Figura \ref{fig:cravacao_result2}. No início da operação, durante a fase de peso próprio, a velocidade de descida do condutor reduz gradualmente até se estabilizar por volta dos 5 segundos. A partir desse ponto, iniciam-se os impactos de martelamento.
	
	\begin{figure}[H]
		\centering
		\includegraphics[width=0.8\linewidth]{cravacao/cravacao_result2.pdf}
		\caption{Profundidade alcançada após a fase de martelamento. Atualmente, o condutor está em uma profundidade de aproximadamente 13,50 metros. Fonte: Autores (2025)}
		\label{fig:cravacao_result2}
	\end{figure}
	
	Durante a fase de peso próprio, o revestimento do condutor atinge aproximadamente 9,5 metros de profundidade após 5 segundos, penetrando a primeira camada de solo e adentrando a segunda, que apresenta maior resistência. Normalmente, essa fase resulta em profundidades entre 10 e 15 metros, indicando que o valor observado está próximo do esperado.
	
Embora a simulação tenha abortado após os 13 m, a Figura \ref{fig:historico2} mostra que solo apresentou um comportamento fisicamente consistente.  Ressalta-se que uma adaptação importante foi considerar o condutor com a ponta fechada. A ponta fechada ocorre devido ao fenômeno de "embuchamento", no qual um \textit{plug} de solo argiloso se forma logo no início da cravação, ocupando a secção transversal do condutor devido à adesão da argila em suas paredes internas. Com isto, foi possível considerar o condutor como sendo uma estrutura maciça, com raio de 18" (0,4572 m). Essa medida foi necessária para melhorar a transição de malha computacional do revestimento para o ambiente externo, elevando a qualidade dos elementos de malha.
	
Para garantir a precisão do modelo na previsão da interação solo-estrutura, foi realizada uma calibração com dados operacionais, conforme ilustrado na Figura \ref{fig:cravacao_result3}.

\begin{figure}[H]
	\centering
	\includegraphics[width=0.7\linewidth]{cravacao/cravacao_result3.pdf}
	\caption{Número acumulado de impactos ao longo da profundidade alcançada somente durante a fase de penetração do martelo. Fonte: Autores (2025)}
	\label{fig:cravacao_result3}
\end{figure}

A Figura \ref{fig:cravacao_result3} compara os dados operacionais da indústria (curva azul) com os resultados simulados (curva vermelha), evidenciando um comportamento semelhante entre ambos. Ajustes futuros no modelo e refinamentos nas simplificações aplicadas podem melhorar ainda mais essa correspondência, conforme será visto mais adiante, nos resultados atuais. 
	
	
	
	\section{Simulação da cravação do revestimento condutor por martelamento}
	
	Com base nos conhecimentos adquiridos a partir das simulações anteriores, foi possível avançar para uma abordagem mais refinada, resultando no desenvolvimento de um novo modelo para a simulação da cravação do revestimento condutor por martelamento.
	
	\subsection{Metodologia}
	
	Assim, a geometria foi implementada utilizando um modelo 2D axisimétrico. O domínio foi discretizado em uma malha não estruturada composta por 4.884 elementos triangulares de três nós, 2.539 nós e 13.805 pontos materiais, distribuídos de forma que a região do solo contém seis pontos materiais por elemento, enquanto a região do condutor possui um por elemento. 
	
	A geometria completa possui 124,68 metros de altura e 29,26 metros de raio. O domínio do solo argiloso é segmentado em quatro camadas distintas: a primeira se estende até 8 metros de profundidade, a segunda até 10 metros, e a terceira e quarta atingem 17 e 25 metros, respectivamente. Para reduzir o custo computacional nas análises iniciais, o solo foi limitado a uma profundidade de 60 metros. O condutor, por sua vez, possui um diâmetro externo de 36 polegadas (~0,91 metros) e um comprimento total de 58,8 metros. A adesão da argila na seção transversal forma um tampão que veda o diâmetro interno do revestimento, permitindo que ele seja tratado como uma estrutura maciça com um peso linear de 3.470 lb/ft. Essa consideração reduziu a necessidade de refinamento excessivo na região próxima ao condutor, facilitando o processo de discretização da malha.
	
	Conforme ilustrado na Figura \ref{fig:Dominio_cracavao}, o revestimento condutor encontra-se inicialmente posicionado acima da mudline. Para melhorar a estabilidade numérica, foi implementada uma ponta no condutor, conforme abordado em estudos anteriores (\citet{ceccato2017adhesive}; \citet{galavi2017numerical}; \citet{galavi2018numerical}; \citet{martinelli2021investigation}; \citet{martinelli2022explicit}; \cite{yost2023addressing}; \citet{ceccato2024simulating}).
	
	As condições de contorno foram definidas da seguinte forma: o movimento horizontal é restrito nas laterais, o deslocamento vertical é limitado no topo, e a fronteira inferior é fixa para impedir qualquer deslocamento. O condutor pode se mover exclusivamente na direção vertical. Além disso, o procedimento de malha móvel está ilustrado na Figura \ref{fig:Dominio_cracavao}, onde a malha comprimida é restrita à região do solo, enquanto a malha móvel ocupa a parte superior, tendo o condutor como referência para seu deslocamento. O revestimento foi modelado como um corpo rígido, e a interação solo-estrutura foi representada por meio do algoritmo de contato adesivo proposto por \citet{alkafaji2013}.
	
	\begin{figure}[H]
		\centering
		\includegraphics[width=0.7\linewidth]{cravacao/Dominio_cravacao.pdf}
		\caption{Geometria em camadas, distribuição da malha e geometria do condutor de um modelo numérico 2D axissimétrico. Fonte: Autores (2024)}
		\label{fig:Dominio_cracavao}
	\end{figure}
	
	\subsubsection{Propriedades dos materiais}
	
	A definição adequada das propriedades dos materiais e dos parâmetros do modelo constitutivo é essencial para garantir a representatividade das simulações numéricas. Assim, de acordo com os dados de resistência à compressão foi possível classificar o solo como uma argila muito mole, considerando-se o critério adotado por \citet{maragon2008}.
	
	Os dados de ensaio CPTu utilizados são disponibilizados pela operadora, a partir dos quais constatou-se que o modelo constitutivo de Mohr-Coulomb seria adequado para modelar a resposta do solo. A Tabela \ref{tab:propriedades_crav} contém as propriedades dos materiais considerados para modelar o solo coesivo não drenado em questão.
	
	\begin{table}[H]
		\centering
		\footnotesize % Reduz tamanho da fonte
		\renewcommand{\arraystretch}{1.2} % Aumenta espaçamento entre linhas
		\setlength{\tabcolsep}{4pt} % Ajusta espaçamento entre colunas
		\begin{tabular}{|cccccc|}
			\hline
			\multicolumn{1}{|c|}{\multirow{2}{*}{\textbf{Propriedades}}} &
			\multicolumn{4}{c|}{\textbf{Valor}} &
			\multirow{2}{*}{\textbf{Unidade}} \\ \cline{2-5}
			\multicolumn{1}{|c|}{} &
			\multicolumn{1}{c|}{Camda 1} &
			\multicolumn{1}{c|}{Camada 2} &
			\multicolumn{1}{c|}{Camada 3} &
			\multicolumn{1}{c|}{Camada 4} &
			\\ \hline
			\multicolumn{1}{|c|}{\textbf{Classificação}} &
			\multicolumn{1}{c|}{Argila muito mole} &
			\multicolumn{1}{l|}{Argila muito mole} &
			\multicolumn{1}{l|}{Argila mole} &
			\multicolumn{1}{l|}{Argila média} &
			\multicolumn{1}{l|}{} \\ \hline
			\multicolumn{1}{|c|}{\textbf{Porosidade inicial}} &
			\multicolumn{1}{c|}{0.58} &
			\multicolumn{1}{c|}{0.58} &
			\multicolumn{1}{c|}{0.47} &
			\multicolumn{1}{c|}{0.43} &
			- \\ \hline
			\multicolumn{1}{|c|}{\textbf{Densidade}} &
			\multicolumn{1}{c|}{1,475.40} &
			\multicolumn{1}{c|}{1,687.91} &
			\multicolumn{1}{c|}{1,799.32} &
			\multicolumn{1}{c|}{1,838.82} &
			Kg/m³ \\ \hline
			\multicolumn{1}{|c|}{\textbf{Coeficiente de Poisson \linebreak efetivo}} &
			\multicolumn{1}{c|}{0.49} &
			\multicolumn{1}{c|}{0.49} &
			\multicolumn{1}{c|}{0.45} &
			\multicolumn{1}{c|}{0.40} &
			- \\ \hline
			\multicolumn{1}{|c|}{\textbf{Valor de K0}} &
			\multicolumn{1}{c|}{0.96} &
			\multicolumn{1}{c|}{0.96} &
			\multicolumn{1}{c|}{0.82} &
			\multicolumn{1}{c|}{0.67} &
			- \\ \hline
			\multicolumn{1}{|c|}{\textbf{Módulo de elasticidade efetivo}} &
			\multicolumn{1}{c|}{17,896.07} &
			\multicolumn{1}{c|}{27,823.35} &
			\multicolumn{1}{c|}{60,440.32} &
			\multicolumn{1}{c|}{83,865.38} &
			kPa \\ \hline
			\multicolumn{1}{|c|}{\textbf{Coesão efetiva}} &
			\multicolumn{1}{c|}{9,27} &
			\multicolumn{1}{c|}{36.23} &
			\multicolumn{1}{c|}{93.95} &
			\multicolumn{1}{c|}{148.92} &
			kPa \\ \hline
			\multicolumn{1}{|c|}{\textbf{Ângulo de atrito efetivo}} &
			\multicolumn{1}{c|}{36.29} &
			\multicolumn{1}{c|}{37.99} &
			\multicolumn{1}{c|}{40.35} &
			\multicolumn{1}{c|}{37.63} &
			graus \\ \hline
			\multicolumn{1}{|c|}{\textbf{Ângulo de atrito (contato)}} &
			\multicolumn{1}{c|}{0.43} &
			\multicolumn{1}{c|}{0.44} &
			\multicolumn{1}{c|}{0.47} &
			\multicolumn{1}{c|}{0.44} &
			rad \\ \hline
			\multicolumn{1}{|c|}{\textbf{Fator de adesão}} &
			\multicolumn{1}{c|}{4.64} &
			\multicolumn{1}{c|}{18.12} &
			\multicolumn{1}{c|}{46.97} &
			\multicolumn{1}{c|}{74.46} &
			kPa \\ \hline
			\multicolumn{6}{|c|}{\textbf{Água}} \\ \hline
			\multicolumn{1}{|c|}{\textbf{Densidade}} &
			\multicolumn{4}{c|}{1000} &
			Kg/m³ \\ \hline
			\multicolumn{1}{|c|}{\textbf{Permeabilidade intrínseca}} &
			\multicolumn{4}{c|}{1.0214x10-9} &
			m³/s \\ \hline
			\multicolumn{1}{|c|}{\textbf{Módulo volumétrico}} &
			\multicolumn{4}{c|}{2.15x104} &
			kPa \\ \hline
			\multicolumn{1}{|c|}{\textbf{Viscosidade dinâmica}} &
			\multicolumn{4}{c|}{1.5673x10-6} &
			kPa   s \\ \hline
		\end{tabular}
		\caption{Propriedades utilizadas para modelar o solo coesivo não drenado}
		\label{tab:propriedades_crav}
	\end{table}
	
	No modelo, o revestimento condutor possui propriedades do aço é considerado como um corpo rígido, com densidade de 7860 kg/m³.   

	\subsection{Análise da influência do ângulo de dilatância}

Com base nas propriedades definidas para o solo e para o condutor, foram realizadas análises numéricas destinadas a investigar a influência dos parâmetros constitutivos na resposta do sistema. Entre eles, destaca-se o ângulo de dilatância, cuja variação foi utilizada para avaliar seus efeitos no comportamento de penetração e na resistência mobilizada durante a cravação.

O ângulo de dilatância $(\psi)$ foi conceituado como uma medida da expansão volumétrica durante o cisalhamento em materiais granulares. Sua relação com a resistência ao cisalhamento foi formalizada por \citet{rowe1962stress} através da teoria da dilatância por intertravamento de partículas, expressa pela Equação \ref{eq:dilatancia}: 
	
	\begin{equation}
	\frac{\sigma1}{\sigma3}=\left( \frac{1+sin\phi}{1-sin\phi} \right)\bullet \left( \frac{1+sin\psi}{1+sin\psi} \right)
	\label{eq:dilatancia}
	\end{equation}

O ângulo de dilatância, em análises geotécnicas, costuma ser assumido
como constante ao longo do processo de plastificação. O valor $\psi = 0$ indica que
o volume do solo é preservado durante o cisalhamento. Em solos argilosos,
mesmo na presença de sobreconsolidação, a dilatância geralmente é muito
baixa ($\psi$$\sim$ 0). Já em materiais granulares, como areias e cascalhos, com um
ângulo de atrito interno $\varphi>$ 30°, o ângulo de dilatância apresenta forte
dependência do atrito interno. Para solos não coesivos, é comum estimar o
ângulo de dilatância como mostra a Equação \ref{eq:dilatancia2}:

	\begin{equation}
	\psi=\varphi-30°
	\label{eq:dilatancia2}
\end{equation}

Neste estudo a respeito da influência do ângulo de dilatância, foi realizada uma análise numérica do processo de cravação de um revestimento condutor em solo não coesivo, representado como um material granular modelado pelo critério constitutivo de Mohr-Coulomb. O domínio de solo foi definido com propriedades geotécnicas típicas de materiais arenosos, incluindo ângulo de atrito interno, módulo de elasticidade e coesão.

Com o objetivo de avaliar a influência do ângulo de dilatância (($\psi$) no comportamento do solo durante o processo de cravação, foram consideradas diferentes variações desse parâmetro, adotando-se os valores de 0°, 5°, 10°, 15° e 20°. Essa estratégia permitiu analisar a sensibilidade da resposta numérica às distintas intensidades de dilatação, possibilitando identificar seus efeitos tanto no comportamento do solo quanto na interação com o revestimento ao longo da simulação.

A análise dos resultados concentrou-se na influência do ângulo de dilatância na profundidade de penetração do condutor ao longo do tempo(Figura \ref{fig:Figura_dilatancia01}). Observa-se que todas as simulações atingem uma profundidade máxima, porém com taxas de penetração distintas. De modo geral, o aumento do ângulo de dilatância resulta na redução da profundidade alcançada, indicando maior resistência do solo à cravação, como pode ser observado na Figura \ref{fig:Figura_dilatancia01}.

	\begin{figure}[H]
	\centering
	\includegraphics[width=0.7\linewidth]{cravacao/Figura_dilatancia01.pdf}
	\caption{Gráfico comparativo de profundidade atingida com os diferentes ângulos de dilatância (Autores, 2025)}
	\label{fig:Figura_dilatancia01}
\end{figure}

A Figura \ref{fig:Figura_dilatancia02} mostra a distribuição das tensões efetivas no solo durante a cravação do condutor. Observa-se concentração de tensões próximas à ponta e ao entorno do revestimento, indicando as regiões onde a resistência do solo é mobilizada. Uma zona intermediária com gradiente de tensões sugere plastificação e rearranjo das partículas, enquanto áreas mais distantes apresentam menores níveis de tensão por ainda não terem sido significativamente solicitadas. Esses resultados evidenciam a redistribuição de tensões causada pela penetração e reforçam a influência do comportamento dilatante na resposta do maciço.

	\begin{figure}[H]
	\centering
	\includegraphics[width=0.6\linewidth]{cravacao/Figura_dilatancia02.pdf}
	\caption{Magnitude da tensão efetiva ao longo do tempo (Autores, 2025)}
	\label{fig:Figura_dilatancia02}
\end{figure}

Conclui-se que o ângulo de dilatância exerce influência significativa na resposta do solo durante a cravação do revestimento, afetando a profundidade de penetração e distribuição de tensões. Os resultados evidenciam a sensibilidade do modelo numérico a esse parâmetro e reforçam a importância de sua adequada consideração para representar de forma realista a interação solo-estrutura em solos não coesivos.

\subsection{MPM Puro e MPM de integração mista}

O MPM consiste em uma abordagem híbrida Euleriana-Lagrangiana que combina as vantagens de métodos baseados em partículas e em malha. Nesse contexto, o domínio computacional é discretizado por uma malha Euleriana de fundo, enquanto os corpos materiais são representados por partículas Lagrangianas que transportam todas as propriedades constitutivas, como tensão, deformação e variáveis de histórico, movimentando-se através da malha. Essa dupla representação permite tratar grandes deformações de forma natural, evitando problemas de distorção de malha típicos de formulações puramente Lagrangianas, ao mesmo tempo em que preserva o acompanhamento do histórico do material.

A simulação da penetração de um corpo rígido em solo envolve deformações elevadas, situação em que o esquema convencional do MPM pode apresentar instabilidades numéricas associadas ao cruzamento de partículas entre elementos. Quando uma partícula atravessa a fronteira de um elemento, descontinuidades nos gradientes das funções de forma podem gerar forças nodais não físicas, reduzindo a precisão da solução.

Como alternativa para mitigar esse efeito, emprega-se o esquema de integração mista implementado no Anura3D. Nesse procedimento, calcula-se inicialmente uma tensão representativa constante para cada elemento, obtida pela média ponderada das tensões dos pontos materiais nele contidos. Em seguida, as forças internas são avaliadas por meio de integração de Gauss, de forma análoga aos procedimentos clássicos do Método dos Elementos Finitos, proporcionando maior estabilidade e suavidade na distribuição de tensões.

Considerando esse contexto, buscou-se analisar a operação da instalação de revestimento condutor em solos coesivos sob condição não drenada, típicos de ambientes offshore, por meio da comparação entre duas abordagens numéricas: o MPM convencional e o esquema de integração MPM-Mixed, que incorpora integração de Gauss. Assim, teve como objetivo avaliar qual formulação apresentava melhor desempenho e adaptabilidade para a simulação do processo.

A Figura \ref{fig:Figura_mpm01} apresenta a comparação entre os esquemas de integração entre o MPM puro e o MPM-Mixed. Observa-se que o MPM puro gera distribuições de tensões mais irregulares, com flutuações abruptas e zonas localizadas de alta intensidade, indicando a presença de instabilidades numéricas associadas ao cruzamento de células. Em contraste, o esquema MPM-Mixed produz campos de tensão mais suaves e contínuos, com transições progressivas entre regiões de diferentes magnitudes e redução de picos espúrios próximos às fronteiras rígidas. 

	\begin{figure}[H]
	\centering
	\includegraphics[width=0.8\linewidth]{cravacao/Figura_mpm01.pdf}
	\caption{Comparação da magnitude da tensão efetiva (kPa) para: (a) MPM puro; (b) Integração mista de MPM; (c) Profundidade de penetração versus tempo (Autores, 2025).}
	\label{fig:Figura_mpm01}
\end{figure}

Além disso, a análise da evolução da profundidade ao longo do tempo mostra que ambos os métodos atingem a profundidade final semelhante, porém a integração mista apresenta maior eficiência computacional, alcançando o resultado em menor tempo. Esses resultados evidenciam a superior estabilidade e desempenho do esquema MPM-Mixed para a simulação do problema analisado.

	\subsection{Resultados atuais}
	
	Os resultados do modelo numérico desenvolvido mostram uma forte concordância com os dados operacionais reais, validando a aplicação do Método do Ponto Material (MPM) na simulação da instalação de revestimentos de condutores em solos argilosos offshore. A precisão na previsão dos deslocamentos e das distribuições de tensão ao redor do condutor confirma a adequação das formulações empregadas para representar as complexas interações solo-estrutura em ambientes offshore desafiadores.
	
	A análise das distribuições de tensão ao longo das fases de peso próprio e martelamento indicou um aumento significativo da resistência do solo com a profundidade, o que contribui para a estabilização da taxa de penetração do revestimento. Esse comportamento está em conformidade com estudos prévios (\citet{galavi2024mpm}; \citet{martinelli2021investigation}), que também relataram um incremento progressivo da resistência do solo durante a instalação de estruturas similares.
	
	A simulação forneceu insights valiosos sobre o comportamento do solo durante a instalação do revestimento do condutor. A estabilização progressiva da descida durante a fase de peso próprio indica que o modelo representa com precisão a interação dinâmica entre o condutor e o solo circundante. Além disso, a resposta do solo ao início dos impactos de martelamento e o consequente aumento da resistência evidenciam a capacidade do modelo de capturar adequadamente as condições de carga dinâmica. Para uma melhor compreensão dos efeitos desse processo, a Figura \ref{fig:cravacao_result1} apresenta a distribuição de tensão no solo em dois momentos distintos: (a) após a fase de peso próprio e (b) após a fase de martelamento.
	
	\begin{figure}[H]
		\centering
		\includegraphics[width=0.7\linewidth]{cravacao/cravacao_result1.pdf}
		\caption{a) Distribuição da tensão vertical efetiva no final da fase de penetração do peso próprio; b) Distribuição da tensão vertical efetiva no final da fase de penetração do martelo. Fonte: Autores (2025)}
		\label{fig:cravacao_result1}
	\end{figure}
	
	A Figura \ref{fig:cravacao_result1} apresenta a distribuição das tensões verticais efetivas no solo. Uma análise mais detalhada das tensões próximas ao revestimento do condutor revela um aumento significativo na concentração de tensões na região da ponta. Esse comportamento é observado tanto na fase de peso próprio (Fig. \ref{fig:cravacao_result1}a) quanto após a aplicação das cargas de impacto (Fig. \ref{fig:cravacao_result1}b).
	
	Além disso, o perfil de descida do condutor ao longo do tempo pode ser analisado na Figura \ref{fig:cravacao_result2}. No início da operação, durante a fase de peso próprio, a velocidade de descida do condutor reduz gradualmente até se estabilizar por volta dos 5 segundos. A partir desse ponto, iniciam-se os impactos de martelamento.
	
	\begin{figure}[H]
		\centering
		\includegraphics[width=0.8\linewidth]{cravacao/cravacao_result2.pdf}
		\caption{Profundidade alcançada após a fase de martelamento. Atualmente, o condutor está em uma profundidade de aproximadamente 13,50 metros. Fonte: Autores (2025)}
		\label{fig:cravacao_result2}
	\end{figure}
	
	Durante a fase de peso próprio, o revestimento do condutor atinge aproximadamente 9,5 metros de profundidade após 5 segundos, penetrando a primeira camada de solo e adentrando a segunda, que apresenta maior resistência. Normalmente, essa fase resulta em profundidades entre 10 e 15 metros, indicando que o valor observado está próximo do esperado.
	
	Em seguida, a aplicação dos impactos provoca uma descida gradual do condutor. Com base na metodologia de \citet{galavi2019}, adotou-se um intervalo de 1 segundo entre impactos. Após um total de 64 impactos, o condutor alcançou cerca de 13,5 metros de profundidade. O estágio de martelamento é um processo complexo, pois a crescente resistência do solo exige um número progressivamente maior de impactos para cada avanço adicional. Para garantir a precisão do modelo na previsão da interação solo-estrutura, foi realizada uma calibração com dados operacionais da indústria, conforme ilustrado na Figura \ref{fig:cravacao_result3}.
	
	\begin{figure}[H]
		\centering
		\includegraphics[width=0.7\linewidth]{cravacao/cravacao_result3.pdf}
		\caption{Número acumulado de impactos ao longo da profundidade alcançada somente durante a fase de penetração do martelo. Fonte: Autores (2025)}
		\label{fig:cravacao_result3}
	\end{figure}
	
	A Figura \ref{fig:cravacao_result3} compara os dados operacionais da indústria (curva azul) com os resultados simulados (curva vermelha), evidenciando um comportamento semelhante entre ambos. Ajustes futuros no modelo e refinamentos nas simplificações aplicadas podem melhorar ainda mais essa correspondência.
	
	Assim, apesar dos avanços, algumas limitações devem ser consideradas. O modelo 2D axisimétrico, embora reduza custos computacionais, pode não representar completamente os efeitos tridimensionais das interações solo-estrutura, especialmente em solos anisotrópicos ou com variações laterais. A ampliação para um modelo 3D pode fornecer uma análise mais detalhada do processo de instalação.
	
	Além disso, a caracterização do solo baseada em dados CPTu pode apresentar variações conforme o local de instalação, impactando a precisão do modelo. Dessa forma, sua aplicação em diferentes contextos geotécnicos depende diretamente da qualidade e representatividade dos dados de entrada.
	
	\subsection{Considerações finais}
	
	O estudo da cravação do revestimento condutor tem avançado significativamente nas etapas mais recentes, demonstrando que o modelo numérico desenvolvido está alinhado com o problema proposto, desempenhando um papel fundamental na previsão do comportamento do condutor durante a instalação. Nesse contexto, a análise evidenciou a influência das propriedades do solo e das condições de cravação na profundidade final atingida, permitindo ajustes mais precisos nos parâmetros operacionais. No entanto, a modelagem adotada foi limitada a um formato axissimétrico, o que pode restringir a representação completa dos efeitos tridimensionais da interação solo-estrutura, além de outras simplificações relacionadas à etapa de martealmento. Diante disso, estudos futuros devem explorar a ampliação do modelo, visando maior precisão nas simulações e uma compreensão mais detalhada dos fatores que influenciam o processo de cravação.

\chapter{Modelagem Computacional da Instalação de Revestimento Condutor por Jateamento}
\label{chap:jateamento}
\section{Considerações Iniciais}

A necessidade de explorar áreas cada vez mais profundas levou ao aprimoramento das técnicas de perfuração de poços. Entre os principais métodos utilizados para iniciar poços offshore, destacam-se o jateamento e a cravação. O jateamento, em particular, é altamente eficiente na instalação do revestimento condutor em sedimentos de águas profundas, que geralmente apresentam baixa consolidação e diagênese reduzida. Esse método minimiza falhas ao preservar a integridade das formações geológicas superficiais, frequentemente frágeis. Por essa razão, o jateamento é amplamente adotado como a técnica predominante na perfuração offshore (\cite{kan2018field}; \cite{akers2008jetting}).

Para o desenvolvimento deste modelo, foram adotadas diversas metodologias com o objetivo de otimizar os resultados da instalação do revestimento condutor por meio do método de jateamento. Nesse contexto, foram desenvolvidos modelos para compreender o comportamento do solo em condições específicas, utilizando o modelo viscoplástico de Herschel–Bulkley para caracterizar o comportamento reológico do solo marinho argiloso quando submetido às forças de cisalhamento exercidas pelo jato de fluido de perfuração. A seguir, são apresentados os desenvolvimentos metodológicos e computacionais adotados nesta etapa do projeto.



\section{Fundamentação e Evolução da Modelagem Numérica}

A modelagem computacional da instalação do revestimento condutor por jateamento vem sendo progressivamente adotada como alternativa às abordagens puramente empíricas, sobretudo devido às limitações operacionais, logísticas e econômicas associadas a ensaios em escala real. O avanço das ferramentas de Dinâmica dos Fluidos Computacional (CFD) possibilitou representar, com maior nível de detalhe, a interação entre fluido de perfuração, solo marinho e estrutura, permitindo avaliar fenômenos que não são diretamente observáveis em campo (\cite{kan2018field}; \cite{gomes2022modelling}).

As simulações foram conduzidas no software ANSYS Fluent 19.2, amplamente consolidado em aplicações de escoamentos multifásicos e problemas com malha dinâmica. A escolha dessa plataforma permitiu a implementação de modelos reológicos, o acompanhamento explícito das fases e a representação do movimento vertical do condutor ao longo do tempo.

O domínio computacional bidimensional adotado nesta etapa possui 90 m de altura e 80 m de largura, conforme ilustrado na Figura A1. Em estudos anteriores, domínios de menor extensão lateral mostraram-se suscetíveis à influência das condições de contorno, afetando a dissipação de pressão e a evolução do campo de velocidades. Assim, a ampliação do domínio foi necessária para minimizar efeitos numéricos artificiais e garantir que os resultados obtidos refletissem predominantemente os fenômenos físicos associados ao jateamento, e não restrições impostas pelas fronteiras do modelo.

O revestimento condutor foi modelado com 50 m de comprimento e diâmetro externo de 36 polegadas, enquanto a broca apresentou diâmetro de 17,5 polegadas. O bit stick-out, definido como a distância entre a broca e a base do condutor, foi fixado em 0,3 m, e a elevação inicial do conjunto em relação ao fundo do mar foi estabelecida em 0,5. Adicionalmente, foi definida uma linha de monitoramento ao longo do eixo central do condutor, conforme ilustrado na Figura A.1, com o objetivo de rastrear a evolução espacial e temporal de propriedades representativas do processo de jateamento.


\begin{figure}[H]
	\centering
	\includegraphics[width=0.8\linewidth]{jateamento/Jateamento_01.pdf}
	\caption{Configuração geométrica e sistema de monitoramento. (a) Dimensões do domínio de simulação; (b) Linha central de monitoramento. Fonte: Autores (2025)}
	\label{fig:Jateamento_01}
\end{figure}

Com base em estudos prévios reportados na literatura, a fração volumétrica inicial do solo foi assumida como VoF = 1 (\cite{guo2022evaluation}), representando um material completamente saturado e contínuo. Essa hipótese mostrou-se adequada para a análise do processo de jateamento em solos argilosos moles, além de contribuir para a estabilidade numérica das simulações multifásicas.
As propriedades físicas e reológicas do solo utilizadas no modelo computacional foram definidas com base em valores típicos reportados na literatura para argilas marinhas moles e são apresentadas na Tabela A1.


\begin{table}[H]
	\centering
	\begin{tabular}{|l|l|l|}
		\hline
		& \textbf{Solo} & \textbf{Água} \\ \hline
		\textbf{Peso Específico (kg/m³)}               & 2300          & 998.3         \\ \hline
		\textbf{Viscosidade (Pa.s)}                    & -             & 0.001         \\ \hline
		\textbf{Viscosidade de escoamento(Pa.s)}       & -             & -             \\ \hline
		\textbf{Tensão de escoamento (Pa)}             & 40000         & -             \\ \hline
		\textbf{Índice de comportamento do fluido (n)} & 0.1          & -             \\ \hline
		\textbf{Índice de consistência (Pa.s)}         & 188310        & -             \\ \hline
		\textbf{Taxa de escoamento crítica (1/s)}    & 5.11          & -             \\ \hline
	\end{tabular}
	\caption{Parâmetros reológicos dos fluidos. Autores: \citet{gomes2022modelling} \citet{pacheco2020numerical}; \citet{salam2019enhancement}}
	\label{fig:Jat_tabelaprop}
\end{table}

No contexto da dissipação das tensões induzidas pelo jato, os resultados indicaram que a distribuição e a magnitude das tensões no solo são fortemente dependentes do nível de vazão imposto, conforme observado em estudos clássicos de jateamento em solos moles (\cite{beck1991reliable}). Conforme ilustrado na Figura A.2, o perfil de descida ao longo de 30 metros do condutor evidência regiões de intensa fluidização próximas à broca, seguidas por zonas onde ocorre dissipação gradual das tensões ao longo da interface solo-condutor Figura A3.


\begin{figure}[H]
	\centering
	\includegraphics[width=0.8\linewidth]{jateamento/Jateamento_02.pdf}
	\caption{Evolução da fração volumétrica do solo (VoF) durante a descida do revestimento condutor: (a) estado inicial em 0 s; (b) penetração intermediária em 75 s; e (c) profundidade final em 150 s. Fonte: Autores (2025)}
	\label{fig:Jateamento_02}
\end{figure}

\begin{figure}[H]
	\centering
	\includegraphics[width=0.8\linewidth]{jateamento/Jateamento_03.pdf}
	\caption{Imagens da dissipação de pressão em diferentes instantes da operação: (a) 5 minutos; (b) 10 minutos; (c) 15 minutos; e (d) 20 minutos. Fonte: Autores (2025)}
	\label{fig:Jateamento_03}
\end{figure}

Embora esse mixed soil possa apresentar, ao longo do tempo, um ganho percentual elevado de resistência devido ao efeito de setup, seu valor final de resistência ao cisalhamento tende a permanecer significativamente inferior ao do solo intacto, uma vez que o processo se inicia a partir de um estado de resistência muito reduzida (\cite{beck1991reliable}). Tal comportamento possui implicações diretas na estabilidade do condutor e na integridade do sistema durante e após a instalação.

Entretanto, os resultados do modelo CFD indicaram a migração da água para o interior do fluido viscoso que representa o solo, que não corresponde plenamente ao mecanismo físico esperado em solos reais. Essa limitação decorre da representação do solo como um fluido não newtoniano homogêneo, o que pode superestimar a mobilidade relativa da fase líquida no interior da matriz sólida. Dessa forma, esse aspecto será objeto de investigação em etapas futuras do trabalho, visando o aprimoramento da representação física do solo.


\section{Modelagem em desenvolvimento}

Esta seção apresenta a etapa atual de desenvolvimento da modelagem computacional da instalação do revestimento condutor por jateamento. Diferentemente de abordagens anteriores mais simplificadas, a estratégia adotada nesta fase busca aumentar o grau de realismo do modelo por meio da incorporação de funções definidas pelo usuário (User-Defined Functions – UDFs), permitindo a representação explícita de condições operacionais extraídas de dados reais de campo.


\subsection{Metodologia atual}

A metodologia atualmente em desenvolvimento baseia-se na utilização de três UDFs implementadas em linguagem C++, integradas ao solver do ANSYS Fluent, como ilustra na figura A.4, com o objetivo de representar de forma mais fiel a dinâmica do processo de jateamento e da instalação do revestimento condutor. Essas funções foram empregadas para controlar a descida do condutor, a vazão do jato e a resposta reológica do solo, ampliando a complexidade e a aderência física do modelo numérico.

\begin{figure}[H]
	\centering
	\includegraphics[width=0.6\linewidth]{jateamento/Jateamento_04.pdf}
	\caption{Esquema de implementação das funções definidas pelo usuário (UDFs) no ANSYS Fluent.}
	\label{fig:Jateamento_04}
\end{figure}

A cinemática de descida do condutor foi ajustada de acordo com a duração real de um processo de jateamento de Marlim, conforme descrito no Relatório do Jateamento do Revestimento na Figura A.4. Em vez de impor uma velocidade constante arbitrária, a UDF responsável pelo movimento vertical prescreve a descida do condutor em função do tempo total de jateamento observado em campo.

\begin{figure}[H]
	\centering
	\includegraphics[width=0.8\linewidth]{jateamento/Jateamento_05.pdf}
	\caption{Trecho do Relatório de Jateamento do Revestimento referente à duração do processo de jateamento.}
	\label{fig:Jateamento_05}
\end{figure}

De maneira análoga, a vazão do jato utilizada nas simulações também foi extraída de dados operacionais de um processo real de jateamento de marlim. Essa vazão é imposta por meio de uma UDF específica, possibilitando a aplicação de condições de contorno coerentes com a prática operacional e evitando a adoção de valores puramente paramétricos ou idealizados.

\begin{figure}[H]
	\centering
	\includegraphics[width=0.7\linewidth]{jateamento/Jateamento_06.pdf}
	\caption{Tabela extraída do Relatório de Jateamento do Revestimento do campo de Marlim contendo os valores operacionais de vazão do jato.}
	\label{fig:Jateamento_06}
\end{figure}

A terceira UDF foi utilizada para incorporar ao modelo reológico do solo o valor de SUTT (Shear Undrained Ultimate Strength) obtido a partir do teste GT-34, fornecido como dado de entrada para o presente estudo. Esse parâmetro foi acoplado diretamente ao modelo viscoplástico de Herschel-Bulkley, aumentando a consistência geotécnica da modelagem.

\begin{figure}[H]
	\centering
	\includegraphics[width=0.7\linewidth]{jateamento/Jateamento_07.pdf}
	\caption{Tabela de resultados do ensaio geotécnico GT-34 contendo os valores de SUTT.}
	\label{fig:Jateamento_07}
\end{figure}


\subsection{Resultados atuais}

Os resultados atualmente obtidos correspondem à simulação da descida do revestimento condutor ao longo de 7,5 metros, representando a fase inicial do processo de jateamento. Essa etapa permitiu avaliar a resposta do solo às condições operacionais impostas, bem como a influência direta de parâmetros geotécnicos incorporados ao modelo por meio das funções definidas pelo usuário. A Figura A.8 apresenta a evolução da descida do condutor ao longo do tempo durante essa fase inicial da simulação.

\begin{figure}[H]
	\centering
	\includegraphics[width=0.7\linewidth]{jateamento/Jateamento_08.pdf}
	\caption{Evolução temporal da profundidade de descida do revestimento condutor durante a fase inicial do processo de jateamento, correspondente aos primeiros 7,5 m de descida. Fonte: Autores (2025)}
	\label{fig:Jateamento_08}
\end{figure}

A análise concentrou-se na evolução da resistência do solo e na variação da viscosidade aparente, ambas diretamente associadas à modificação do valor de SUTT acoplado ao modelo viscoplástico de Herschel–Bulkley e ao nível de vazão imposto aos jatos de perfuração. Observou-se que alterações nesses parâmetros resultam em mudanças significativas no comportamento reológico do solo, afetando sua resposta ao cisalhamento imposto tanto pelo jato quanto pelo avanço do condutor. A Figura A.9 ilustra a pressão do solo ao longo do tempo.


\begin{figure}[H]
	\centering
	\includegraphics[width=0.7\linewidth]{jateamento/Jateamento_09.pdf}
	\caption{Evolução temporal da pressão do solo durante a fase inicial do processo de jateamento. Fonte: Autores (2025)}
	\label{fig:Jateamento_09}
\end{figure}

Adicionalmente, foi observada a entrada da fase associada ao solo no interior do condutor ao longo da simulação. Esse comportamento sugere a ocorrência de um processo de arraste do material fluidizado para dentro do revestimento, possivelmente relacionado ao nível de vazão imposto no jato. Diante desse resultado, estão sendo conduzidas investigações adicionais com o objetivo de avaliar estratégias para mitigar ou evitar a entrada do solo no interior do condutor, incluindo ajustes nos níveis de vazão, na formulação das funções definidas pelo usuário e na representação numérica do acoplamento entre as fases.

\section{Considerações finais}

O presente trabalho apresentou o estágio atual de desenvolvimento de uma metodologia computacional para a modelagem da instalação de revestimento condutor por jateamento, com foco na incorporação progressiva de maior realismo físico e operacional ao modelo numérico. A utilização de funções definidas pelo usuário permitiu integrar ao ambiente de simulação dados reais de operação e de ensaios geotécnicos, elevando o nível de complexidade e a representatividade do problema analisado.

Apesar dos avanços alcançados, reconhece-se que a modelagem ainda se encontra em desenvolvimento. Limitações inerentes à representação do solo como um meio contínuo viscoplástico permanecem, especialmente no que diz respeito à migração da água no interior do material modelado. Esses aspectos serão objeto de investigações, com vistas ao aprimoramento da representação física do solo e à ampliação do escopo das simulações.


\cleardoublepage
\phantomsection

\addcontentsline{toc}{chapter}{Bibliografia}
\printbibliography

\end{document}
